\documentclass[../Bachelorarbeit.tex]{subfiles}
\begin{document}
\chapter{Analyse \& Recherche}
\label{chap:analyse}
\section{State of the Art}
\label{chap:analyse:sec:sota}

\ideas{Literaturrecherche ... sowie was aktueller Stand der Technik sowie Forschung.}

\subsection{Kriterien der Analyse}
\label{chap:analyse:sec:sota:sec:kriterien_der_analyse}

\ideas{evtl. eine Art Katalog aufstellen und Kriterien def. die für die Analyse (in Bezug auf das Projekt) relevant sind (Bezug zu Themen aus der Einleitung herstellen)}

\subsection{Google Maps}
\label{chap:analyse:sec:sota:sec:google_maps}
\todoImprovement[]{weitere Analyse}
\todoInfo[]{Was ist gut, was ist schlecht?}

\section{Analyse von bestehenden Konzepten}
\label{chap:analyse:sec:analyBestehendeKonz}
\todoImprovement[]{Abschnittstitel konkretisieren}
\todoImprovement[]{Thema genauer ausarbeiten}
\ideas{Anhand des Telefon-Features von Pery (bestehendes Basis-System) eine Analyse für die neuen Features durchführen ... was ist gut ... was ist schlecht (evtl. Kundenumfrage)}

\section{Hardware Recherche}
\label{chap:analyse:sec:hw_recherce}

\ideas{App vs. Einbaugerät ... Probleme und Fragestellungen kommen hier rein ... Arbeit distanzieren von dem Thema - da es sich nicht um das Hauptthema handelt.}

\end{document}