\documentclass[../Bachelorarbeit.tex]{subfiles}
\begin{document}
\chapter{Analyse \& Recherche}
\label{chap:analyse}

Nachdem in Kapitel \ref{chap:einfuehrung} - \nameref{chap:einfuehrung} die ersten konkreten Überlegungen bis hin zu einem \nameref{sec:anwendungsszenario} aufgezeigt wurden um den Inhalt und die Funktionsweise des Prototypen zu umreisen, wird in diesem Kapitel mit der Domäne auseinander setzen für die der Prototyp entwickelt wird.\\
\\
Es werden \nameref{chap:analyse:sec:interviews} durchgeführt um zu analysieren auf welche Art und Weise  Domänenexpert\_innen arbeiten.
Zum einen soll herausgefunden werden, wo sich aktuell Flaschenhälse, in ihrem Workflow befinden und zum anderen, was sie für Wünsche und Anforderungen an ihre Planungswerkzeuge stellen.\\
\\
Darauf hin folgt eine \nameref{chap:analyse:sec:sota}, welche wiederum in zwei Teile unterteilt wird.
Der erste Teil dokumentiert die Erkenntnisse die in der Literaturanalyse gewonnen werden.
Des weiteren beschäftigt sich der zweite Teil mit Software die einen Bezug zum Prototyp aufweist.


\section{State of the Art}
\label{chap:analyse:sec:sota}

\ideas{Literaturrecherche ... sowie was aktueller Stand der Technik sowie Forschung.}

\subsection{Kriterien der Analyse}
\label{chap:analyse:sec:sota:sec:kriterien_der_analyse}

\ideas{evtl. eine Art Katalog aufstellen und Kriterien def. die für die Analyse (in Bezug auf das Projekt) relevant sind (Bezug zu Themen aus der Einleitung herstellen)}

\subsection{Google Maps}
\label{chap:analyse:sec:sota:sec:google_maps}

Daten OSM

\todoImprovement[]{weitere Analyse}
\todoInfo[]{Was ist gut, was ist schlecht?}

\section{Analyse von bestehenden Konzepten}
\label{chap:analyse:sec:analyBestehendeKonz}
\todoImprovement[]{Abschnittstitel konkretisieren}
\todoImprovement[]{Thema genauer ausarbeiten}
\ideas{Anhand des Telefon-Features von Pery (bestehendes Basis-System) eine Analyse für die neuen Features durchführen ... was ist gut ... was ist schlecht (evtl. Kundenumfrage)}

\section{Hardware Recherche}
\label{chap:analyse:sec:hw_recherce}

\ideas{App vs. Einbaugerät ... Probleme und Fragestellungen kommen hier rein ... Arbeit distanzieren von dem Thema - da es sich nicht um das Hauptthema handelt.}

\end{document}