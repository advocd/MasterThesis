\documentclass[../Bachelorarbeit.tex]{subfiles}
\begin{document}
\chapter{Analyse \& Recherche}
\label{chap:analyse}

Nachdem in Kapitel \ref{chap:einfuehrung} - \nameref{chap:einfuehrung} die ersten konkreten Überlegungen bis hin zu einem \nameref{sec:anwendungsszenario} aufgezeigt wurden um den Inhalt und die Funktionsweise des Prototypen zu umreisen, wird in diesem Kapitel mit der Domäne auseinander setzen für die der Prototyp entwickelt wird.
Für diesen Zweck ist das Kapitel in zwei Abschnitte unterteilt. 
Im ersten Abschnitt (\nameref{chap:analyse:sec:sota}) wird ein Blick auf die Forschung und Fachliteratur geworfen während sich der zweite Abschnitt (\nameref{chap:analyse:sec:analyBestehendeKonz}) mit echten Projekten beschäftigt die eine Relevanz für den Prototypen darstellen.


\section{Literaturrecherche}
\label{chap:analyse:sec:sota}

\ideas{Literaturrecherche ... sowie was aktueller Stand der Technik sowie Forschung.}


\ideas{evtl. eine Art Katalog aufstellen und Kriterien def. die für die Analyse (in Bezug auf das Projekt) relevant sind (Bezug zu Themen aus der Einleitung herstellen)}


Daten OSM

\todoImprovement[]{weitere Analyse}
\todoInfo[]{Was ist gut, was ist schlecht?}

\section{Analyse von bestehenden Konzepten}
\label{chap:analyse:sec:analyBestehendeKonz}
\todoImprovement[]{Abschnittstitel konkretisieren}
\todoImprovement[]{Thema genauer ausarbeiten}


\subsection{Google Maps}
\label{chap:analyse:sec:sota:sec:google_maps}

\subsection{Pery}
\ideas{Anhand des Telefon-Features von Pery (bestehendes Basis-System) eine Analyse für die neuen Features durchführen ... was ist gut ... was ist schlecht (evtl. Kundenumfrage)}

\end{document}