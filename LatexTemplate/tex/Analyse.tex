\documentclass[../Bachelorarbeit.tex]{subfiles}
\begin{document}
\chapter{Analyse \& Recherche}
\label{chap:analyse}

\section{Interviews: Workflows}
\label{chap:analyse:sec:interviews}
Dieser Teil beschäftigt sich mit der Fragestellung wie Personen ihre Außendienstlichen Tätigkeiten organisieren, welchen Herausforderungen sie im beruflichen Alltag gegenüberstehen und welche Wünsche und Verbesserungen Sie sich wünschen. 
Für diesen Zweck sollen Interviews und Hand-ons  geführt werden, die sich an einem Leitfaden orientieren. 
Das Ziel dieser Interviews besteht darin, ein besseres Gefühl für den Ist-Zustand zu bekommen und Anhand dieser Erkenntnisse die möglichen Defizite zu analysieren.
Des weiteren bietet dieser Ansatz die Möglichkeit, Verbesserungswünsche und Ideen von Personen aus der Domäne zu erhalten ohne das sie zuvor durch den Blick aus technischer Sicht verfälscht wurden.

\section{State of the Art}
\label{chap:analyse:sec:sota}

\ideas{Literaturrecherche ... sowie was aktueller Stand der Technik sowie Forschung.}

\subsection{Kriterien der Analyse}
\label{chap:analyse:sec:sota:sec:kriterien_der_analyse}

\ideas{evtl. eine Art Katalog aufstellen und Kriterien def. die für die Analyse (in Bezug auf das Projekt) relevant sind (Bezug zu Themen aus der Einleitung herstellen)}

\subsection{Google Maps}
\label{chap:analyse:sec:sota:sec:google_maps}
\todoImprovement[]{weitere Analyse}
\todoInfo[]{Was ist gut, was ist schlecht?}

\section{Analyse von bestehenden Konzepten}
\label{chap:analyse:sec:analyBestehendeKonz}
\todoImprovement[]{Abschnittstitel konkretisieren}
\todoImprovement[]{Thema genauer ausarbeiten}
\ideas{Anhand des Telefon-Features von Pery (bestehendes Basis-System) eine Analyse für die neuen Features durchführen ... was ist gut ... was ist schlecht (evtl. Kundenumfrage)}

\section{Hardware Recherche}
\label{chap:analyse:sec:hw_recherce}

\ideas{App vs. Einbaugerät ... Probleme und Fragestellungen kommen hier rein ... Arbeit distanzieren von dem Thema - da es sich nicht um das Hauptthema handelt.}

\end{document}