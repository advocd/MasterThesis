\documentclass[Bachelorarbeit.tex]{subfiles}
\begin{document}
\chapter{Interviews}

\newpage
\section{Leitfaden für Interviews}
Da die Aussagen bei den Gesprächen vermutlich sehr unterschiedlich ausfallen werden wurde kein expliziter Fragenkatalog entworfen. Vielmehr soll der Leitfaden eine Orientierung für das Gespräch darstellen und somit den groben Rahmen definieren.

\begin{enumerate}
	\item Allgemeine Angaben
	\begin{enumerate}
		\item Datum und Dauer des Interviews:
		\begin{itemize}
			\item[] 
		\end{itemize}
		\item Umfeld in dem das Interview geführt wird:
		\begin{itemize}
			\item[]
		\end{itemize}
	\end{enumerate}
	\item Angaben zur Person
	\begin{enumerate}
		\item Alter:
		\begin{itemize}
			\item[] 
		\end{itemize}
	\end{enumerate}
	\item Angaben zum Unternehmen
	\begin{enumerate}
		\item Selbstbezeichnung durch Proband\_in  (\ac{KMU}, internationaler Konzern, etc.):
		\begin{itemize}
			\item[] 
		\end{itemize}
		\item Tätigkeitsfeld des Unternehmens:
		\begin{itemize}
			\item[] 
		\end{itemize}
	\end{enumerate}
	\item Angaben zur Funktion im Unternehmen
	\begin{enumerate}
		\item Tätigkeit im Unternehmen:
		\begin{itemize}
			\item[] 
		\end{itemize}
		\item Verantwortungsgrad der Planung:
		\begin{enumerate}
			\item[] 
		\end{enumerate}
		\item Zuständigkeitsbereich:
		\begin{itemize}
			\item[] 
		\end{itemize}
	\end{enumerate}
	\item Ablauf des Standard Planungs-Workflows (Schritt für Schritt):
	\begin{itemize}
		\item[] 
	\end{itemize}
	\item Sonderfälle des Planungs-Workflows zeigen/erklären lassen (Schritt für Schritt):
	\begin{itemize}
		\item[] 
	\end{itemize}
	\item Probleme und Engpässe des Planungs-Workflows:
	\begin{itemize}
		\item[] 
	\end{itemize}
	\item Gewünschte Verbesserung (aus Domänen-Sicht):
	\begin{itemize}
		\item[] 
	\end{itemize}
\end{enumerate}
\newpage

\section{Interview I}
\begin{enumerate}
	\item Allgemeine Angaben
	\begin{enumerate}
		\item Datum und Dauer des Interviews:
		\begin{itemize}
			\item[] 19.04.2016 - ca. 35 min.
		\end{itemize}
		\item Umfeld in dem das Interview geführt wird:
		\begin{itemize}
			\item[] Das Interview wurde spontan im Zuge eines Besuchs im Firmensitz (Perfany) geführt.
		\end{itemize}
	\end{enumerate}
	\item Angaben zur Person
	\begin{enumerate}
		\item Alter:
		\begin{itemize}
			\item[] ca. 30-35 Jahre
		\end{itemize}
	\end{enumerate}
	\item Angaben zum Unternehmen
	\begin{enumerate}
		\item Selbstbezeichnung durch Proband\_in  (\ac{KMU}, internationaler Konzern, etc.):
		\begin{itemize}
			\item[] Nationaler Konzern mit Niederlassung in Bregenz
		\end{itemize}
		\item Tätigkeitsfeld des Unternehmens:
		\begin{itemize}
			\item[] Dienstleister in der Arbeitskräftevermittlung
		\end{itemize}
	\end{enumerate}
	\item Angaben zur Funktion im Unternehmen
	\begin{enumerate}
		\item Tätigkeit im Unternehmen:
		\begin{itemize}
			\item[] Ausschließlich im Außendienst
		\end{itemize}
		\item Verantwortungsgrad der Planung:
		\begin{enumerate}
			\item[] Selbständig Planung
		\end{enumerate}
		\item Zuständigkeitsbereich:
		\begin{itemize}
			\item[] Bundesland Vorarlberg
		\end{itemize}
	\end{enumerate}
	\item Ablauf des Standard Planungs-Workflows (Schritt für Schritt):
	\begin{itemize}
		\item Es handelt sich um Wiederkehrende Termine
		\item Es wird im Vorfeld für jede Kalenderwoche ein zu betreuender Bezirk gewählt und dieser im Kalender dokumentiert.
		\item Es wird nach Möglichkeit der Termin in eine Woche gelegt die für den Bezirk definiert wurde in dem sich die Niederlassung des Kunden befindet.
	\end{itemize}
	\item Sonderfälle des Planungs-Workflows zeigen/erklären lassen (Schritt für Schritt):
	\item[] Sonderfall: Terminverschiebung von Kundenseite
	\label{interview1:sonderfall}
	\begin{enumerate}
		\item Termin fällt in richtige Wochen-Bezirks-Konstellation
		\label{interview1:sonderfall_optimal}
		\begin{enumerate}
			\item freien Termin-Slot finden, evtl. leichte Umplanung
			
		\end{enumerate}
		\item Termin fällt nicht in richtige Wochen-Bezirks-Konstellation
		\begin{enumerate}
			\item Termin kann auf die nächste korrekte Wochen-Bezirks-Konstellation verlegt werden:
			siehe \ref{interview1:sonderfall_optimal}
			\item Termin kann nicht verlegt werden:
			\begin{enumerate}
				\item[] je nach Abweichung des Bezirks entsteht entsprechender Mehraufwand durch die Anfahrt
			\end{enumerate}
		\end{enumerate}
	\end{enumerate}
	\item Probleme und Engpässe des Planungs-Workflows:
	\begin{itemize}
		\item Urlaubsvertretungen
		\item neuer Kundenkontakt: muss Eingeschoben werden (siehe: \ref{interview1:sonderfall})
	\end{itemize}
	\item Gewünschte Verbesserung (aus Domänen-Sicht):
	\begin{itemize}
		\item[] Optimierung der Route durch Kartenansicht
	\end{itemize}
\end{enumerate}
\newpage

\section{Interview II}
\begin{enumerate}
	\item Allgemeine Angaben
	\begin{enumerate}
		\item Datum und Dauer des Interviews:
		\begin{itemize}
			\item[] 27.04.2016 ca. 90 min.
		\end{itemize}
		\item Umfeld in dem das Interview geführt wird:
		\begin{itemize}
			\item[] Konferenz via Skype
		\end{itemize}
	\end{enumerate}
	\item Angaben zur Person
	\begin{enumerate}
		\item Alter: 
		\begin{itemize}
			\item[] ca. 40-45 Jahre
		\end{itemize}
	\end{enumerate}
	\item Angaben zum Unternehmen
	\begin{enumerate}
		\item Selbstbezeichnung durch Proband\_in  (\ac{KMU}, internationaler Konzern, etc.):
		\begin{itemize}
			\item[] \ac{KMU} mit Sitz in Wien
		\end{itemize}
		\item Tätigkeitsfeld des Unternehmens:
		\begin{itemize}
			\item[] Vertrieb von Hifi-Geräten für den professionellen Einsatz in Tonstudios etc. 
		\end{itemize}
	\end{enumerate}
	\item Angaben zur Funktion im Unternehmen
	\begin{enumerate}
		\item Tätigkeit im Unternehmen:
		\begin{itemize}
			\item[] Geschäftsführer und Außendienst im eigenen Unternehmen.
		\end{itemize}
		\item Verantwortungsgrad der Planung:
		\begin{enumerate}
			\item[] Selbständig Planung
		\end{enumerate}
		\item Zuständigkeitsbereich:
		\begin{itemize}
			\item in erster Linie Österreich 
			\item Ausnahmen: EU und Russland  (Portugal, Schweden,  Moskau)
		\end{itemize}
	\end{enumerate}
	\item Ablauf des Standard Planungs-Workflows (Schritt für Schritt):
	\begin{enumerate}
		\item Route wird definiert - bsp. Süd Österreich
		\item PLZ auf der Route werden zusammengetragen
		\item Kunden werden im System nach PLZ sortiert. Problem: PLZ sind nicht immer direkte Nachbarn.
		\item Ergebnis wird weiter gefiltert nach diversen Metriken (Umsatz, Datum letzte Bestellung, etc.)
		\item Adressen der gefilterten Kunden werden Exportiert.
		\item Adressen werden für Routenberechnung in Google Maps importiert
		\item Für jeden Termin wird ein Post-It mit kundenspezifischen Daten (Adresse, Öffnungszeiten, Umsatz, Datum letzter Verkauf) angefertigt
	\end{enumerate}
	\item Sonderfälle des Planungs-Workflows zeigen/erklären lassen (Schritt für Schritt):
	\begin{enumerate}
		\item Planungsphase
		\begin{itemize}
			\item Eigentlich fixe Touren (Süd-Österreich) gewisse Flexibilität benötigt. Wie beispielsweise Abweichung (Kunden die nicht auf der fixen Route liegen) 
		\end{itemize}
		\item Im Außendienst
		\begin{itemize}
			\item Kunde fällt aus: Welcher Kunde ist in der Nähe von dem aktuellen Standpunkt
			\item neue Kunden einschieben: Durch Empfehlungen von Bestandskunden. 
		\end{itemize}
	\end{enumerate}
	\item Probleme und Engpässe des Planungs-Workflows:
	\begin{itemize}
		\item Effizienzsteigerung - Außendienstler soll beim Kunden sein und nicht im Büro am planen
		\item Mit bestehenden Softwarelösungen: entweder Umsatzdaten oder \ac{GIS} 
	\end{itemize}
	\item Gewünschte Verbesserung (aus Domänen-Sicht):
	\begin{itemize}
		\item Keine automatisch berechneten Vorschläge vom System. Vielmehr Unterstützung durch (Meta-)Daten und Visualisierung
		\item Zusätzliche Meta-Information bei Außendiensttätigkeit wie beispielsweise:
		\begin{itemize}
			\item Interessen des Kunden/Smalltalk-Themen
			\item Berichte über Verkaufte Artikel und mögliche ergänzende Artikel
			\item Top 5 Produkte (nach Umsatz und nach Stückzahl)
		\end{itemize}
		\item Reihung der Route soll dynamisch. änderbar sein (Bsp. Stau, Verschiebung, etc)
		\item Kartenansicht von Kundenstandorten mit wichtigen Metriken über die Kunden (Umsatz, Datum der letzten Bestellung, etc.)
		\item 
	\end{itemize}
\end{enumerate}
\newpage
\section{Interview III}
\begin{enumerate}
	\item Allgemeine Angaben
	\begin{enumerate}
		\item Datum und Dauer des Interviews:
		\begin{itemize}
			\item[] 29.04.2016, ca. 60 min.
		\end{itemize}
		\item Umfeld in dem das Interview geführt wird:
		\begin{itemize}
			\item[] Im privaten Umfeld
		\end{itemize}
	\end{enumerate}
	\item Angaben zur Person
	\begin{enumerate}
		\item Alter:
		\begin{itemize}
			\item[] 
		\end{itemize}
	\end{enumerate}
	\item Angaben zum Unternehmen
	\begin{enumerate}
		\item Selbstbezeichnung durch Proband\_in  (\ac{KMU}, internationaler Konzern, etc.):
		\begin{itemize}
			\item[] 
		\end{itemize}
		\item Tätigkeitsfeld des Unternehmens:
		\begin{itemize}
			\item[] 
		\end{itemize}
	\end{enumerate}
	\item Angaben zur Funktion im Unternehmen
	\begin{enumerate}
		\item Tätigkeit im Unternehmen:
		\begin{itemize}
			\item[] 
		\end{itemize}
		\item Verantwortungsgrad der Planung:
		\begin{enumerate}
			\item[] 
		\end{enumerate}
		\item Zuständigkeitsbereich:
		\begin{itemize}
			\item[] 
		\end{itemize}
	\end{enumerate}
	\item Ablauf des Standard Planungs-Workflows (Schritt für Schritt):
	\begin{itemize}
		\item[] 
	\end{itemize}
	\item Sonderfälle des Planungs-Workflows zeigen/erklären lassen (Schritt für Schritt):
	\begin{itemize}
		\item[] 
	\end{itemize}
	\item Probleme und Engpässe des Planungs-Workflows:
	\begin{itemize}
		\item[] 
	\end{itemize}
	\item Gewünschte Verbesserung (aus Domänen-Sicht):
	\begin{itemize}
		\item[] 
	\end{itemize}
\end{enumerate}

\end{document}