\documentclass[Bachelorarbeit.tex]{subfiles}
\begin{document}
\chapter{Interviews}

\newpage
\section{Leitfaden für Interviews}
Da die Aussagen bei den Gesprächen vermutlich sehr unterschiedlich ausfallen werden wurde kein expliziter Fragenkatalog entworfen. Vielmehr soll der Leitfaden eine Orientierung für das Gespräch darstellen und somit den groben Rahmen definieren.

\begin{comment}
1. Allgemeine Angaben
a. Datum  
b. Dauer
2. Angaben zum Kontext
a. Wo wurde das Interview geführt
b. Gab es eine Einführung/Einleitung zum Interview
c. Welche Software/Medien (analog/digital) werden verwendet
3. Angaben zur Person
a. Alter
4. Angaben zum Unternehmen:
a. Selbstbezeichnung durch Proband_in  (KMU, ...)
b. ca. Anzahl der Mitarbeiter_innnen
5. Angaben zur Funktion im Unternehmen
a. Tätigkeit im Unternehmen
b. Planungsgrad der Tätigkeit:
i. Selbständig Planung, vorgegebene Grobplannung, etc.
c. Zuständigkeitsbereich
i. ... auf Stadt-, Bezirks-, Landes-, etc.

6. Ablauf des Standard Planungs-Workflows zeigen/erklären lassen (Schritt für Schritt)
7. Sonderfälle des Planungs-Workflows zeigen/erklären lassen (Schritt für Schritt)
8. Probleme und Engpässe des Planungs-Workflows zeigen/erklären lassen
9. Welche Verbesserung sind gewünscht (aus Domänen-Sicht)

\end{comment}

\begin{enumerate}
	\item Allgemeine Angaben
	\begin{enumerate}
		\item Datum, Dauer
	\end{enumerate}
	\item Angaben zum Kontext
	\begin{enumerate}
		\item Wo wurde das Interview geführt
		\item Gab es eine Einführung/Einleitung zum Interview
		\item Welche Software/Medien (analog/digital) werden verwendet
	\end{enumerate}
	\item Angaben zur Person
	\begin{enumerate}
		\item Alter
	\end{enumerate}
	\item Angaben zum Unternehmen
	\begin{enumerate}
		\item Selbstbezeichnung durch Proband\_in  (KMU, ...)
		\item ca. Anzahl der Mitarbeiter\_innen im Unternehmen
		\item Tätigkeitsfeld des Unternehmens
	\end{enumerate}
	\item Angaben zur Funktion im Unternehmen
	\begin{enumerate}
		\item Tätigkeit im Unternehmen
		\item Planungsgrad der Tätigkeit:
		\begin{enumerate}
			\item Selbständig Planung, vorgegebene Grobplannung, etc.
		\end{enumerate}
		\item Zuständigkeitsbereich
		\begin{enumerate}
			\item auf Stadt-, Bezirks-, Landesebene, etc.
		\end{enumerate}
	\end{enumerate}
	\item Ablauf des Standard Planungs-Workflows zeigen/erklären lassen (Schritt für Schritt)
	\item Sonderfälle des Planungs-Workflows zeigen/erklären lassen (Schritt für Schritt)
	\item Probleme und Engpässe des Planungs-Workflows zeigen/erklären lassen
	\item Welche Verbesserung sind gewünscht (aus Domänen-Sicht)
\end{enumerate}
\newpage

\section{Interview I}
\newpage
\section{Interview II}
\newpage
\section{Interview III}
\newpage

\chapter{Diagramme und Bilder}
\label{chap:diagramme_und_bilder}




\section{Übersicht}
\begin{itemize} 
\item \nameref{sec:mock_ups}
\begin{itemize}
\item TEST
\end{itemize}



\end{itemize}

\newpage
\chapter*{Mock-Ups}
\label{sec:mock_ups}


\end{document}