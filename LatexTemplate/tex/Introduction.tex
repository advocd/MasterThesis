\documentclass[Bachelorarbeit.tex]{subfiles}
\begin{document}
\chapter{Einführung}
\label{chap:einfuehrung}

Das Ziel der Arbeit ist zu untersuchen, 
inwiefern die Ergänzung  von Informationen mit geografischen Daten, 
zu einer Optimierung von Entscheidungen beiträgt und welcher Bedeutung dabei der Darstellungsform zukommt. 
Für diesen Zweck wird, im Zuge der Arbeit, ein Prototyp entwickelt, der die Anwender\_innen bei der Planung von Außendienstrouten unterstützen soll. 
Dabei sollen mit dem Prototypen nicht klassische Probleme der Informatik oder Logistik wie Beispielsweise das \textit{travelings salesman problem} gelöst werden. Vielmehr soll den Anwender\_innen ein Werkzeug zur Verfügung gestellt werden, dass ihnen vernetzte Informationen übersichtlich zur Verfügung stellt um sinnvolle Entscheidungen treffen zu können.

\section{Problemstellung}
\label{chap:einfuehrung:sec:problemstellung}

Bei diversen Gesprächen mit Kunden\_innen traten, in regelmäßigen Abständen, immer wieder die Nachfrage für eine Softwarelösung auf, welche die Planung und Organisation von Außendiensttätigkeiten vereinfachen soll. 
Was auf den ersten Blick trivial erscheinen mag, wirkt nach kurzen Überlegungen durchaus interessant. 
Beispielsweise in der ersten Phase (Planung) müssen Routen erstellt werden, die in der Praxis aus diversen Datenquellen oder gar unterschiedlichen Medien stammen. 
Des Weiteren fließen in die Planung, kundenspezifische Erfahrungswerte sowie lokale Ortskenntnisse mit ein.
Zusätzlich hält die Außendiensttätigkeit, auch noch nach der Planung, weitere interessante Herausforderungen bereit, diese reichen von der Unterstützung bei der Durchführung außer Hause,  bis hin zu der Nachbearbeitung bei die neuen Daten ins System eingepflegt werden müssen.
Anhand dieser vielfältigen und teils komplexen sowie verteilten Informationen sollen möglichst effiziente Routen geplant werden. 


\begin{comment}
Als Grundlage für den Prototypen dient die Software Pery der Firma Perfany GmbH. 
Dabei handelt es sich um 
Wobei der Fokus auf dem Ticket Modul der Software ruht. Dabei ist das Anwendungskonzept des Moduls so ausgelegt das sämtliche Aufgaben, die die Firma betreffen , einzeln als Tickets erfasst werden. 
\end{comment}

\section{Idee}
\label{chap:einfuehrung:sec:idee}

Basierend auf der \nameref{chap:einfuehrung:sec:problemstellung} soll sich an dieser Stelle platz für die ersten eigenen Überlegungen finden. 
Diese Ideen stellen noch keine endgültige Lösung da, sondern sollen als Orientierung dienen welche progressiv durch kommende Analysen, \nameref{chap:analyse:sec:interviews} und Feedbacks angepasst werden.\\
\\
Die grundlegende Idee besteht darin, Informationen (Ressourcen und Aufgaben) mit geografischen Daten zu verknüpfen und diese zu visualisieren und somit die Nutzer\_innen bei ihren Entscheidungsprozessen zu unterstützen. 
Dabei soll auf die beiden Bereiche: \textit{Datenfilterung/-Anreicherung} sowie \textit{Darstellungsformen} besonderes Augenmerk gelegt werden.

\subsection*{Datenfilterung/-Anreicherung}
Ein wichtiger Aspekt, um die Unterstützung durch den Prototypen, bei der Planung, zu maximieren besteht darin, dass die richtigen Informationen, zum richtigen Zeitpunkt, am richtigen Ort zur Verfügung stehen. 
Ein Beispiel für die Datenanreicherung ist, wie zuvor erwähnt, die Verwendung von geographischen Informationen und nimmt in dieser Arbeit einen zentralen Punkt ein. 
Dabei dienen die angereicherten Daten als Grundlage für die Visualisierung.
Inwiefern es zu dem Filtern beziehungsweise Anreichern der Daten kommt wird sich Anhand der Analyse, über die Bedürfnisse (siehe Kapitel.: \ref{chap:analyse:sec:interviews} - \nameref{chap:analyse:sec:interviews}) der Anwender\_innen, herausstellen.

\subsection*{Darstellungsformen}
Die Visualisierung der Daten steht in keinster Weise in einer untergeordneten Rolle.
Erst durch den sinnvollen Einsatz der Darstellung, werden die vorhandenen Daten zu einem wichtigen Indikator bei der Entscheidungsfindung.
Dabei versucht die Arbeit sich kritisch mit der Optimierung der Darstellung auseinander zusetzen und diese in späteren Versuchen am Prototyp zu evaluieren.
Ein Beispiel dafür wäre die Thematik einer Darstellung im Kartenformat.
Seit dem Erfolg von Google Maps werden zunehmend Kartenansichten bei der Darstellung von geografischen Daten eingesetzt. 
Diese Arbeit wird sich unter anderem mit der Frage auseinander setzen, welche Planungs-Szenarios, beziehungsweise Workflow-Schritte, durch eine Karten- und/oder Listenansicht besser unterstützt werden.
Um dies herauszufinden wird, im späteren Verlauf, eine Analyse mit Hilfe des Prototypen durchgeführt.
Eine weitere Überlegung besteht darin, die Anwender\_innen selbst entscheiden zu lassen, welche Ansicht sie für welchen Zweck bevorzugen und die Ergebnisse der Analyse als änderbare Standarteinstellung zu verwenden.


\section{Hintergrund}
\label{chap:einfuehrung:sec:hintergrund}
Als Grundgerüst für den Prototypen dient die Software \textit{pery} der Firma \textit{Perfany GmbH}, welche um die unten genannten Anwendungsfälle erweitert werden soll. 
\textit{pery} ist eine webbasierte (Software as a Service) \ac{ERP} sowie \ac{CRM} Lösung.
Das Ziel von \textit{pery} besteht darin, die eigenen Firmendaten mit einander zu verknüpfen um die alltägliche Arbeit im Büro zu erleichtern.
Beispielsweise öffnet sich bei den Anwender\_innen ein Popup sobald ein Anruf eingeht. 
Wenn es sich dabei um einen im System bestehende\_n Kunde\_in handelt kann direkt auf das Popup geklickt werden und eine Partnerübersicht öffnet sich.
In dieser Partnerübersicht finden sich relvanten Informationen (offene Rechnungen, Stammdaten und vieles mehr) sowie weiterführende Links zu diversen History-Elementen dieser Geschäftsbeziehung.
Zusätzlich kann, über eine Tastenkombination, eine globale Suche aufgerufen werden um diverse Entitäten anhand von Namen oder Attributswerten zu finden. \\
\\
Wie im Abschnitt \nameref{chap:einfuehrung:sec:problemstellung} besprochen werden verschiedenste Informationen für die Planung benötigt die teilweise in verschiedenen Systemen liegen. 
Für die Lösung des Problems bezüglich den verteilten Daten gibt es Grundsätzlich zwei Ansätze.
Zum einen würde die Möglichkeit bestehen eine Software zu erstellen, welche sich ihre Daten, über Schnittstellen, aus verschiedensten Quellen\footnote{Diese wären beispielsweise eine externe \ac{API} und/oder Software von Drittanbietern.} holt.
Zum anderen kann eine bestehende Lösung verwendet werden, die um die gewünschte Funktionalität erweitert wird und somit, nach einer importieren der Daten, die ehemaligen Systeme ablöst. 
Da Pery über eine gute Vernetzung der Kundendaten verfügt und das Hauptaugenmerk dieser Arbeit, nicht auf die Implementierung der grundlegender Infrastruktur des Prototypen liegt, wird im Rahmen dieser Arbeit der zweite Fall behandelt (erweitern einer bestehenden Lösung).
\newpage


\section{Anwendungsszenario}
Nachdem in den Abschnitten \nameref{chap:einfuehrung:sec:problemstellung} und \nameref{chap:einfuehrung:sec:idee} erste grundlegende Gedanken umrissen wurden, soll an dieser Stelle aufgezeigt werden, wie möglicherweise die spätere Verwendung des Prototypen ablaufen könnte.
Anhand des Anwendungsszenarios, lässt sich zum einen, durch die Darstellung an einem Beispiel, dass Konzept besser verstehen und zum anderen, fallen grobe Konzept- oder Logikfehler so schon frühzeitig auf.
Wie in den vorhergehenden Abschnitten dieses Kapitels schon angedeutet handelt es sich auch in diesem noch nicht um ein fertiges Konzept sondern um eine Grundlage, die im Laufe der Entwicklung weiter angepasst wird.

\subsection{Szenario: On Trip Information }
Ziel dieses Szenarios ist es, einen speziellen Anwendungsfall für die Verwendung des Prototypens zu konstruieren, anhand dessen später unter anderem evaluiert werden kann, in welcher Form sich die zusätzlichen Informationen (geografischen Daten) auf die Entscheidungsfindung auswirken.
Bei diesem Fall handelt es sich um eine Optimierung für die Planung und Durchführung von Außendiensteinsätzen. 

\paragraph*{Fiktiver Hintergrund}
Um das Beispiel realistischer und verständlicher Wirken zu lassen wird die Handlung in einen fiktiven Rahmen eingebettet. 
Im Verlauf dieser Arbeit wird der fiktive Hintergrund durch detaillierter Personas
\footnote{
	Mehr Informationen was eine Persona ist und wie diese zustande gekommen ist: siehe Abs.: \ref{persona} - \nameref{persona}
	} 
\footnote{
	Damit die Persona's realistischer sind, wurde die Entscheidung getroffen,
	sie auf den gewonnen Erfahrungen der \nameref{chap:analyse:sec:interviews} aufzubauen.
	} 
abglöst.\\
\\
Babsi Zimmermann, 36 Jahre alt, ist eine aufstrebende Mitarbeiterin in der Firma \textit{Purple Circle} und dort als Verkäuferin angestellt. 
Die Firma \textit{Purple Circle} mit Sitz in Dornbirn/Österreich sieht ihr Profession im Sondermaschinenbau und hat sich in diesem Marktsegment durch ihre enge Kundenbindung und qualitativ hochwertige Arbeit einen Namen gemacht. 

\paragraph*{Vorbedingungen}
Im ersten Schritt sollten vor der Planung ungefähre Kriterien für die zur Auswahl stehenden Möglichkeiten definiert sein.
Diese könnten betriebswirtschaftliche Faktoren, wie Verkaufszahlen oder Umsatz sein, aber auch aus dem Bereich des \ac{CRM} stammen, wie beispielsweise: Dauer seit dem letzten Kundenbesuch.
Mithilfe der geografischen Informationen lassen sich zusätzlich auch Kriterien wie maximale Entfernungen auswählen

\paragraph*{Planung}
Babsi öffnet über das Menü die Planungsansicht und wählt die Firma \textit{Rieden} als als Ausgangspunkt für die Planung aus.
Kurz darauf zeigt ihr das System weitere Informationen entlang der Route und am Ziel auf einer Karte an.
Nachdem sie keine dringenden Termine für den Tag hat, entschließt sie sich, die Zeit zu nehmen, Kunden zu besuchen bei denen der letzte Besuch schon längerer Zeit zurückliegt.
Dafür ändert sie dementsprechend die Einstellungen für die Auswahlkriterien auf \textit{Dauer seit letzten Besuch} wodurch die Farbcodierung der Kundenmarker auf der Karte angepasst werden.
Kurz darauf kommt ihre Mitarbeiterin Sylvia vorbei und bittet Babsi ihr ein paar offene Tickets abzunehmen. 
Durch die Umstellung der Filterfunktion werden auf ihrer Karte nun zusätzlich die aktuellen Tickets angezeigt.
Sie sieht das zwei Tickets auf ihrer Route liegen, durch einen klick auf den Marker des ersten Tickets öffnet sich ein Popup über den Marker das ihr den Titel und die Kurzbeschreibung anzeigt. 
Da sie beim ersten Ticket schon Bescheid weiß, um was es sich handelt, klickt sie direkt im Popup auf den Button und übernimmt es in ihre Auswahlliste.
Die Informationen des zweiten Tickets sagen ihr leider nicht soviel, aber das ist kein Problem da sie durch einen Klick auf den Titel direkt, in einem neuen Fenster, auf die Übersichtsseite des entsprechenden Tickets gelangt und dort alle Informationen zu dem Ticket sieht die sich im System befinden.
Zurück in der Kartenansicht übernimmt sie auch das zweite Ticket. 
Dabei fällt ihr auf das in der nähe des zweiten Tickets noch ein roter Kundenmarker ist.
Mit einem Klick auf den roten Kundenmarker öffnet sich wieder ein Popup und sie erfährt das bei der Firma \textit{ZornTec} schon seit sieben Monaten keine Betreuung mehr stattgefunden hat.



\paragraph*{Ablauf: Planung}
\textbf{Beispiel: Außendienst Mitarbeiter\_in}
\begin{enumerate}
	\item Mitarbeiter\_in wählt Ziel der Route aus 
	\begin{enumerate}
		\item Ziel kann Ticket, Kunde oder Adresse sein
	\end{enumerate}
	\item System zeigt weitere Informationen entlang der Route oder am Ziel an\footnote{Die Auswahl der Informationen soll gefiltert werden können. Eventuell mehrere Filter Ebenen wie Kundenbetreuung oder offene Tickets. Auf Basis der getroffenen Filterebene können anschließend weitere Filter gewählt werden wie beispielsweise: geplanter Zeitaufwand von offenen Ticket, aktueller Betreuung Status, etc. }
	\begin{enumerate}
		\item Mögliche Informationen:
		\begin{enumerate}
			\item Offene Tickets
			\item Betreuungsstatus von Kunden\footnote{Betreuungstatus: ist ein Schlüssel der sich aus: Betreuungsaufwand, Priorität des Kunden und Dauer seit dem letzten Betreuungstermin zusammensetzt.}
			\item Evtl. weitere Informationen
		\end{enumerate}
	\end{enumerate}
	\item Mitarbeiter\_in wählt zusätzliche Aufgaben aus
	\begin{enumerate}
		\item System weißt das Ticket der/dem Mitarbeiter\_in zu
		\item Evtl. automatisch weitere Tickets anlegen und der/dem Mitarbeiter\_in zuweist.\footnote{Beispiel: Betreuungstermin – System legt automatisch ein Ticket zur Termins-findung/-vereinbarung mit dem Kunden an und weißt es der/dem Mitarbeiter\_in zu.}
	\end{enumerate}
\end{enumerate}

\begin{comment}

\subsection{Usecase I - Enhanced Ticket (\ac{UC}1)}
Ein wichtiger Bestandteil des (bestehenden) Systems besteht darin Tickets zu verwalten.\footnote{automatische Erstellung, anlegen sowie anderen Mitarbeiter\_innen zuweisen}
Dieses Feature wird verstärkt von \ac{KMU}’s mit Schwerpunkt auf außendienstlichen Tätigkeiten eingesetzt. 
Rückmeldungen von diesen Nutzer\_innen Gruppen hat ergeben, dass der Prozess der Ticket Zuteilung an Mitarbeiter\_innen Optimierungspotential hat. \\
\\

\textbf{Beispiel: IT-Dienstleister}
\begin{enumerate}
	\item Kunde des Dienstleisters erstellt neues Ticket
		\begin{enumerate}
			\item Geo-Daten werden an das Ticket angefügt
		\end{enumerate}
	\item Dispatcher des Dienstleisters reagiert auf Ticket
		\begin{enumerate}
			\item Einstufung der Priorität\footnote{Vorschläge durch das System (Stammdaten - Priorität des verknüpften Kunden) – Auswahl basiert auf der Entscheidung des Dispatcher}
			\item Ressourcen ermitteln:
			\begin{enumerate}
				\item Welche/r Mitarbeiter\_in ist verfügbar und geografisch am nächsten (Anfahrtswegoptimierung)\footnote{Fragestellung: Visualisierung der Ergebnisse }
				\item Ist kein/e Mitarbeiter\_in verfügbar: Vorschläge vom System welcher Mitarbeiter von aktueller Aufgabe abgezogen werden kann (bsp.: interne Aufträge)
			\end{enumerate}
			\item Ticket auf Resource zuweisen\footnote{Resource (Mitarbeiter\_in) und Kunde werden informiert}
		\end{enumerate}
	\item Zugewiesene/r Mitarbeiter\_in hat Ticket gelöst
	\begin{enumerate}
		\item Resourcen wurden vom System auf den Auftrag verbucht
		\item Ticket wird abgeschlossen
	\end{enumerate}
\end{enumerate}

\subsection{Usecase II - On Trip Information (\ac{UC}2)}
Hierbei handelt es sich um weiteres Feature für die Optimierung von Planung- bzw. Arbeitsvorbereitungs- Prozessen von Außendienst Mitarbeitern. Diese sollen bei der Planung ihrer Route, durch das einblenden zusätzlicher Information, unterstützt werden. \\
\\
\textbf{Beispiel: Außendienst Mitarbeiter\_in}
\begin{enumerate}
	\item Mitarbeiter\_in wählt Ziel der Route aus 
	\begin{enumerate}
		\item Ziel kann Ticket, Kunde oder Adresse sein
	\end{enumerate}
	\item System zeigt weitere Informationen entlang der Route oder am Ziel an\footnote{Die Auswahl der Informationen soll gefiltert werden können. Eventuell mehrere Filter Ebenen wie Kundenbetreuung oder offene Tickets. Auf Basis der getroffenen Filterebene können anschließend weitere Filter gewählt werden wie beispielsweise: geplanter Zeitaufwand von offenen Ticket, aktueller Betreuung Status, etc. }
	\begin{enumerate}
		\item Mögliche Informationen:
		\begin{enumerate}
			\item Offene Tickets
			\item Betreuungsstatus von Kunden\footnote{Betreuungstatus: ist ein Schlüssel der sich aus: Betreuungsaufwand, Priorität des Kunden und Dauer seit dem letzten Betreuungstermin zusammensetzt.}
			\item Evtl. weitere Informationen
		\end{enumerate}
	\end{enumerate}
	\item Mitarbeiter\_in wählt zusätzliche Aufgaben aus
	\begin{enumerate}
		\item System weißt das Ticket der/dem Mitarbeiter\_in zu
		\item Evtl. automatisch weitere Tickets anlegen und der/dem Mitarbeiter\_in zuweist.\footnote{Beispiel: Betreuungstermin – System legt automatisch ein Ticket zur Termins-findung/-vereinbarung mit dem Kunden an und weißt es der/dem Mitarbeiter\_in zu.}
	\end{enumerate}
\end{enumerate}

\end{comment}

\end{document}

