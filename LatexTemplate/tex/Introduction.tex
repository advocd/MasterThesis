\documentclass[Bachelorarbeit.tex]{subfiles}
\begin{document}
\chapter{Einführung}
\label{chap:einfuehrung}

Die Arbeitsdokumentation stellt für Organisationen einen wichtigen Faktor bei der Qualitätssicherung dar. 
Um aussagekräftige Analysen erstellen zu können, müssen die gesammelten Daten korrekt und konsistent sein. 
Leider zeigt die Erfahrung,  dass vor allem im Umfeld von Vereinen und gemeinnützigen Organisationen nur selten einheitlichen Methoden zum Erfassen vorhanden sind und/oder diese nicht sachgemäß verwendet werden. 
In Zeiten von Smartphones liegt der Gedanke nahe, die aktuelle Technologie zu verwenden um den Arbeitsaufwand der Dokumentation zu minimieren. \\
\\
Damit eine breite Masse als Zielgruppe angesprochen werden kann sollte das Programm für die geläufigsten mobilen Betriebssystem verfügbar sein. 
Im Moment besitzen die beiden konkurrierenden Betriebssystemen Android und iOS die größten Marktanteile im deutschsprachigen Raum. 
Allerdings versucht Microsoft mit Windows Mobile immer aggressiver Marktanteile zu gewinnen. 
Da diese Plattformen grundsätzlich nicht miteinander kompatibel sind, muss die Applikation für jedes Betriebssystem eigens implementiert werden. 
Die Nachteile liegen auf der Hand, neben dem Entwicklungsaufwand wird auch die Wart- und Erweiterbarkeit eingeschränkt.  \\
\\
Um diese Problematik zu entschärfen gibt es verschiedene Lösungsansätze. 
Zum einen kann das Programm in zwei Teile aufgeteilt werden. 
Bei diesem Vorgehen besteht die Applikation aus einen plattformunabhängigen- und einen plattformspezifischen Teil. 
Hierbei liegt das Ziel darin, den plattformspezifischen Code so weit wie möglich zu minimieren, wodurch der Grad der Wiederverwertbarkeit für andere Plattformen gesteigert wird. 
Zum anderen existieren verschiedene Frameworks für die Multiplattform-Entwicklung. 
Durch den Einsatz dieser Frameworks soll das Programm nur einmal für alle unterstützen Plattformen entwickelt werden. Bei diesem Vorgehen programmiert der Entwickler gegen die plattformunabhängigen Schnittstellen des jeweiligen Frameworks.  \\
\\
Grundsätzlich haben beide Ansätze Stärken und Schwächen, welche je nach den Bedürfnissen des Projektes anders zu gewichten sind. 
Das Ziel dieser Arbeit ist es, die verschiedenen Möglichkeiten zu analysieren und anhand einer Beispiel-Applikation für die Plattformen Android und Windows Phone8 zu implementieren.
Auf die Berücksichtigung der iOS-Plattform wird in dieser Arbeit verzichtet.
Diese Entscheidung basiert darauf, dass die Thematik "`Entwicklung für Android- und iOS-Plattformen"' aus der Sicht des Autors bereits häufig in Fachzeitschriften oder dem Internet diskutiert und dokumentiert wurde\footnote{Ein Anzeichen dafür zeigt das Ergebnis einer Internet-Suche. Der Term: "`android and ios cross platform development"' (14.700.000 Treffer) ist wesentlich häufiger zu finden als: "`android and windows phone cross platform development"' (5.270.000 Treffer) }. 
\\

\section{Projektbeschreibung}
\label{sec:projektbeschreibung}

Im Zuge einer Kooperation mit der \ac{OJAD} soll eine Applikation entwickelt werden, welche das Personal beim Dokumentieren von Veranstaltungen unterstützen soll. 
Bei dem Projekt soll es sich um eine verteilte Anwendung handeln. 
Die MitarbeiterInnen haben die Möglichkeit Veranstaltungen - für die Sie verantwortlich sind - mit Hilfe ihrer Smartphones zu dokumentieren. \footnote{Die Kriterien der unterstützten Endgeräte können dem Abschnitt: \nameref{sec:anforderungen} entnommen werden.}
\\
Bei der Dokumentation handelt es sich in erster Linie um das erfassen von statistischen BesucherInnen-Werten wie beispielsweise Altersgruppen und Geschlecht. 
Diese Werte sollen zentral gespeichert werden und als Grundlage für statistische Auswertungen dienen.
\footnote{Das erzeugen oder erstellen der statistischen Auswertung ist nicht Bestandteil dieser Arbeit beziehungsweise des hier dokumentierten Projektes.}
\\
Ein weiterer Aspekt liegt in der Erfassung des Veranstaltungsablaufes. Neben Fotos ist an dieser Stelle auch Platz für Kommentare vorgesehen. 
Durch die Verwendung der Kommentar-Funktion besteht die Möglichkeit ein persönliches Feedback in die Dokumentation einfließen zu lassen. 
\newpage
 

\section{Anforderungen}
\label{sec:anforderungen}
Ziel dieses Abschnittes ist es, die Rahmenbedingungen für das Projekt zu definieren. 
Diese Punkte dienen als Grundlage für Entscheidungen welche im späteren Verlauf des Projektes auftreten.
Um einen gewissen Spielraum für die folgende \nameref{chap:state_of_the_art}-Analyse sowie die
spätere \nameref{chap:entwicklung} zu ermöglichen, werden an dieser Stelle die Anforderungen nur auf das Nötigste begrenzt.

\subsection*{Applikationen}
Ziel ist es eine Applikation zu entwickeln welche auf den zwei definierten Smartphone-Plattformen lauffähig ist.
Auf Wunsch der \ac{OJAD} wurde definiert, dass die Plattformen Windows Phone 8 sowie Android ab der Version 4.0 - Ice Cream Sandwich(\ac{API} 14) verwendet werden sollen.
In der ersten Phase der Entwicklung, welche in dieser Arbeit dokumentiert wird, soll ein lauffähiger Prototyp für jede Plattform entstehen. 
Der Abschnitt: \nameref{sec:usecases} des \ref{chap:entwicklung}. Kapitels definiert den Funktionsumfang der Applikation.
Dieser Prototyp muss noch nicht den Status der Serienreife erreichen, sondern dient der \ac{OJAD} als Ausgangspunkt für die weitere Entwicklung beziehungsweise Integration in ein größeres Projekt. 

\subsection*{Kommunikation}
Für die Kommunikation zwischen dem Endgerät und dem Server soll die \ac{JSON} eingesetzt werden. Dabei handelt es sich um ein kompaktes Datenformat, welches sich in der Kommunikation mit Webservices etabliert hat. 
Aus Gründen die im Abschnitt: \nameref{subsec:sicherheit} angeführt werden, wird keine Verschlüsselung wie Beispielsweise \ac{SSL} verwendet.

\subsection*{Webservice}
Als Schnittstelle zwischen Server und Applikation soll ein Webservice verwendet werden. 
Dieser wird nach dem \ac{REST}-Konzept realisiert. 
Durch die Verwendung des \ac{REST}-ful Webservice soll die Applikation Zugriff auf die vom Server bereitgestellten Funktionen erhalten.  
Desweiteren soll - für beide Plattformen - der gleiche Webservice genutzt werden.

\subsection*{Server}
Da die Infrastruktur der \ac{OJAD} auf Microsoft-Produkte ausgelegt ist, liegt die Entscheidung nahe, den Server auf diesen Technologien aufzubauen. Für die aktuelle Entwicklung wird eine lokale MS-SQL Datenbank eingesetzt. 
Als Betriebssystem für den Produktiv-Server wurde ein Windows-Server 2012R2 in der 64-Bit-Version ausgewählt. 
Dieser verfügt über ein Intel Xeon 2,33 GHz-Prozessor sowie zwei GB-RAM.



\subsection*{Sicherheit}
\label{subsec:sicherheit}

Aus Sicht der \ac{OJAD} werden durch die aktuell geplante Funktionalität der Applikation keine sensiblen Daten über das Internet übertragen. 
Deshalb wird - in Absprach mit der \ac{OJAD} - keine Verschlüsselung für die Kommunikation verwendet.
In den anfänglichen Gesprächen mit der \ac{OJAD} ist der Wunsch aufgekommen, dass ausschließlich die verantwortlichen MitarbeiterInnen eine ihnen zugeteilte Veranstaltung dokumentieren dürfen. 
Für diesen Zweck wurden die Sicherheitsziele: Authentifizierung (Pin) sowie Autorisierung (nur verantwortliche MitarbeiterInnen dürfen dokumentieren) definiert.


\end{document}