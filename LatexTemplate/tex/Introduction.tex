\documentclass[Bachelorarbeit.tex]{subfiles}
\begin{document}
\chapter{Einführung}
\label{chap:einfuehrung}

Das Ziel dieses Kapitels ist es, zum einen die Idee des Projektes zu transportieren und zum anderen den groben Rahmen des Projektes zu skizzieren.
Dabei handelt es sich in diesem Kapitel in erste Linie um die Gedanken des Autors, die vor den umfangreichen Analysen und Recherchen (siehe Kap.: \ref{chap:analyse} - \nameref{chap:analyse}) getätigt wurden.
Dabei wird die Arbeit, im Speziellen die Entwicklung des Prototypen, durch die Erkenntnisse der progressiven Schritte in den jeweiligen Kapiteln stets weiterentwickelt und damit auch die vorausgegangenen Ideen und Konzepte.


\section{Motivation und Hintergrund}
\label{chap:einfuehrung:sec:hintergrund}

Das Ziel dieser Arbeit besteht darin zu untersuchen ob mittels des Einsatz von alternativen Darstellungsformen und kontextsensitiver Daten der Entscheidungsfindungsprozess vereinfacht werden kann. 
Dafür soll konkret für den Anwendungsfall der Außendienstplanung ein Konzept ausgearbeitet und in Form eines Prototypen realisiert werden. \\
\\
Diese Arbeit ist in Kooperation mit dem Unternehmen Perfany GmbH entstanden.
Die Firma Perfany, mit Sitz in Lustenau/Österreich wurde 2011 von den beiden Geschäftsführern Christian Rhomberg und Andreas Zwerger gegründet.
Die Aufgabenbereiche von Perfany reichen von der Beratung wie auch Betreuung von individuellen IT-Systemen bis hin zur Software Entwicklung von Pery, in welcher auch der Autor tätig ist. 
Dabei ist Pery für diese Arbeit von Bedeutung da es als Grundgerüst dienen soll auf dem der Prototyp aufbaut wird.\\
\\
Bei Pery handelt es sich um eine webbasierte (Software as a Service) \ac{ERP} sowie \ac{CRM} Lösung.
Das Ziel von Pery besteht darin, die eigenen Firmendaten miteinander zu verknüpfen, um die alltägliche Arbeit im Büro zu erleichtern.
An folgendem Beispiel soll verdeutlicht werden, was mit dem Verknüpfen der Firmendaten in Pery gemeint ist.\\
\\
Sobald ein Anruf in der Telefonanlage\footnote{Die Telefonanlage muss mit Pery kompatibel und eingebunden sein.} eingeht, öffnet sich ein Popup, welches die wichtigsten Informationen des Anrufs anzeigt. 
Wenn es sich dabei um bestehende Kunden\_in handelt, kann direkt auf das Popup geklickt werden, um eine Partnerübersicht zu öffnen.
In dieser Partnerübersicht finden sich relevante Kundeninformationen (offene Rechnungen, Stammdaten und vieles mehr) sowie weiterführende Links zu diversen History-Elementen dieser Geschäftsbeziehung.
Zusätzlich kann über eine Tastenkombination eine globale Suche aufgerufen werden, um diverse Entitäten anhand von Namen oder Attributswerten zu finden. \\
\\
Ein weiteres Merkmal von Pery ist die progressive Weiterentwicklung im direkten Kontakt mit den Kunden\_innen.
Für diesen Zweck werden regelmäßig Interviews und Feedbackgespräche durchgeführt.
Anhand dieser Gespräche ist aufgefallen, dass eine rege Nachfrage für eine Softwarelösung besteht, welche die Planung und Organisation von Außendiensttätigkeiten vereinfachen soll.
Aus diesem Wunsch heraus ist die Idee entstanden, zu untersuchen, inwiefern geografische Informationen den Entscheidungsfindungsprozess beeinflussen können.


\section{Problemstellung}
\label{chap:einfuehrung:sec:problemstellung}
Die Problemstellung stellt den ersten analytischen Schritt der Entwicklung des Prototypen da.
Hierbei soll analysiert werden, was für Probleme im aktuellen Ist-Zustand bestehen.
Stellvertretend für das Thema der Arbeit bezieht sich die Problemstellung auf den Anwendungsfall zur Planung einer Außendienstroute.
Aus meiner Sicht bestehen dabei folgende drei Problembereiche: verteilte/isolierte Informationen, Komplexität sowie Wissensmanagement.

\paragraph*{Verteilte/isolierte Informationen}
Die für die Planung relevanten Informationen sind aufgrund von fehlenden Lösungen auf verschiedene Systeme oder Medien verteilt und isoliert. 
Um eine Planung durchzuführen, müssen die benötigten Information separat aus den verschiedenen Systemen oder Medien geholt werden.
Dabei dienen teilweise die Ergebnisse aus einem System als Grundlage für die Suche in weiteren Systemen.
Dies ist zum einen aufwendig und zeitraubend sowie fehleranfällig, da einzelne Informationen übersehen werden können.
\newpage

\paragraph*{Komplexität}
Verschiedenste kundenspezifische Attribute\footnote{
	Diese können zum einen betriebswirtschaftliche Faktoren sein wie der Umsatz des letzten Jahres und zum anderen aus dem Aufgabenbereich des \ac{CRM} stammen wie beispielsweise die Dauer seit dem letzten Besuch.
	} dienen als Grundlage für die Planung.
Meist reichen dabei einzelne Attribute nicht aus sondern es wird eine Matrix aus Informationen benötigt.
Dabei erhöht sich die Komplexität mit jeder Information die hinzugefügt wird.
Aufgrund der Komplexität steigt zum einen die Fehleranfälligkeit und zum anderen der benötigte Aufwand bei der Planung.

\paragraph*{Wissensmanagement}
In die meisten Planungen fließen kundenspezifische Erfahrungswerte sowie lokale Ortskenntnisse mit ein.
Dabei besteht das Problem, dass dieses Wissen nicht für Dritte\footnote{Ein klassisches Beispiel sind neue Kräfte im Unternehmen und/oder Urlaubsvertretungen} verfügbar ist.
Als mögliche Folgen können beispielsweise ineffiziente oder gar fehlerhafte Planung sowie ein erhöhtes Zeitaufkommen bei der Planung entstehen. 


\begin{comment}
Als Grundlage für den Prototypen dient die Software Pery der Firma Perfany GmbH. 
Dabei handelt es sich um 
Wobei der Fokus auf dem Ticket Modul der Software ruht. Dabei ist das Anwendungskonzept des Moduls so ausgelegt das sämtliche Aufgaben, die die Firma betreffen , einzeln als Tickets erfasst werden. 
\end{comment}

\section{Idee}
\label{chap:einfuehrung:sec:idee}

Die grundlegende Idee besteht nun darin, Informationen (Ressourcen und Aufgaben) mit geografischen Daten zu verknüpfen und diese zu visualisieren und somit die Nutzer\_innen bei ihren Entscheidungsprozessen zu unterstützen. 
Dabei liegt das Augenmerk auf den folgenden Bereichen: Datenstruktur, Datenfilterung/-Anreicherung sowie Darstellung, welche aufeinander aufbauend sind. 


\subsection*{Datenstruktur}
Im ersten Schritt sollte eine zielführende Datenstruktur entwickelt werden, welche das Grundgerüst für die beiden anderen Bereiche darstellt und ihre Funktionalität gewährleistet.
Wie in Abschnitt \nameref{chap:einfuehrung:sec:problemstellung} besprochen, werden verschiedenste Informationen für die Planung benötigt, die teilweise in verschiedenen Systemen liegen. 
Für die Lösung des Problems bezüglich den verteilten Daten gibt es grundsätzlich zwei Ansätze.
Zum einen besteht die Möglichkeit eine Zwischenschicht zu entwickeln, welche sich ihre Daten über Schnittstellen aus verschiedensten Quellen\footnote{Diese wären beispielsweise externe \ac{API} und/oder Software von Drittanbietern.} holt und in den Prototypen einpflegt.
Zum anderen kann das bestehende System (Pery), welches um die gewünschte Funktionalität erweitert wird, die bestehenden Drittsysteme ablösen. 
Da Pery über eine gute Vernetzung der Kundendaten verfügt und das Hauptaugenmerk dieser Arbeit nicht auf der Implementierung der grundlegenden Infrastruktur des Prototypen liegt, wird im Rahmen dieser Arbeit der zweite Fall behandelt (erweitern einer bestehenden Lösung).

\subsection*{Datenfilterung/-Anreicherung}
Ein wichtiger Aspekt, um die Unterstützung durch den Prototypen, bei der Planung zu maximieren, besteht darin, dass die richtigen Informationen zum richtigen Zeitpunkt am richtigen Ort zur Verfügung stehen. 
Ein Beispiel für die Datenanreicherung ist, wie zuvor erwähnt, die Verwendung von geographischen Informationen, dies nimmt in dieser Arbeit einen zentralen Punkt ein. 
Dabei dienen die angereicherten Daten als Grundlage für die Visualisierung.
Inwiefern es zu dem Filtern beziehungsweise Anreichern der Daten kommt, wird sich Anhand der Analyse über die Bedürfnisse (siehe Kapitel.: \ref{chap:analyse:sec:interviews} - \nameref{chap:analyse:sec:interviews}) der Anwender\_innen herausstellen.

\subsection*{Darstellungsformen}
Die Visualisierung der Daten steht in keinster Weise in einer untergeordneten Rolle.
Durch den sinnvollen Einsatz der Darstellung werden die vorhandenen Daten zu einem wichtigen Indikator bei der Entscheidungsfindung (vgl. \cite{Reiterer2000}, S. 1f).
Deswegen versucht die Arbeit sich kritisch mit der Optimierung der Darstellung auseinander zusetzen und diese in späteren Versuchen am Prototyp zu evaluieren.
Ein Beispiel dafür wäre die Thematik einer Darstellung im Kartenformat.
Seit dem Erfolg von Google Maps werden zunehmend Kartenansichten bei der Darstellung von geografischen Daten eingesetzt (vgl. \cite{Mitchell2008}, S. 8). 
Des weiteren wird sich die Arbeit unter anderem mit der Frage auseinander setzen, welche Planungs-Szenarios, beziehungsweise Workflow-Schritte, durch eine Karten- und/oder Listenansicht besser unterstützt werden.\\
\\
Um dies herauszufinden wird, im späteren Verlauf, eine Analyse mit Hilfe des Prototypen durchgeführt.
Eine weitere Überlegung besteht darin, die Anwender\_innen selbst entscheiden zu lassen, welche Ansicht sie für welchen Zweck bevorzugen und die Ergebnisse der Analyse als änderbare Standardeinstellung zu verwenden.


\section{Anwendungsszenario}
\label{sec:anwendungsszenario}
Nachdem in den Abschnitten \nameref{chap:einfuehrung:sec:problemstellung} und \nameref{chap:einfuehrung:sec:idee} erste grundlegende Gedanken umrissen wurden, soll an dieser Stelle aufgezeigt werden, wie möglicherweise die spätere Verwendung des Prototypen ablaufen könnte.\\
\\
Anhand des Anwendungsszenarios lässt sich durch die Darstellung an einem Beispiel zum einen das Konzept besser verstehen und zum anderen fallen grobe Konzept- oder Logikfehler so schon frühzeitig auf.
Wie in den vorhergehenden Abschnitten dieses Kapitels schon angedeutet, handelt es sich noch nicht um ein fertiges Konzept, sondern um eine Grundlage, die im Laufe der Entwicklung weiter angepasst wird.

\subsection{Szenario: On Trip Information }
Ziel dieses Szenarios ist es, einen speziellen Anwendungsfall für die Verwendung des Prototypen zu konstruieren. 
Anhand des Prototypen soll später unter anderem evaluiert werden, in welcher Form sich die zusätzlichen Informationen (geografischen Daten) auf die Entscheidungsfindung auswirken.
In diesem Fall handelt es sich um eine Optimierung für die Planung und Durchführung von Außendiensteinsätzen.
Damit das Beispiel realistischer und verständlicher erscheint, wird die Handlung in folgenden fiktiven Rahmen eingebettet:\\

Babsi Zimmermann, 36 Jahre alt, ist eine aufstrebende Mitarbeiterin der Firma Purple Circle und dort als Verkäuferin angestellt. 
Die Firma Purple Circle, mit Sitz in Dornbirn/Österreich, sieht ihre Profession im Sondermaschinenbau und hat sich in diesem Marktsegment durch ihre enge Kundenbindung und qualitativ hochwertige Arbeit einen Namen gemacht. 


\paragraph*{Vorbedingungen}
Im ersten Schritt sollten vor der Planung ungefähre Kriterien für die zur Auswahl stehenden Möglichkeiten definiert sein.
Diese könnten betriebswirtschaftliche Faktoren wie Verkaufszahlen oder Umsatz sein, aber auch aus dem Bereich des \ac{CRM} stammen, wie beispielsweise die Dauer seit dem letzten Kundenbesuch.
Mithilfe der geografischen Informationen lassen sich zusätzlich auch Kriterien wie maximale Entfernungen auswählen.

\paragraph*{Ablauf Planung}
Babsi öffnet die Planungsansicht und wählt die Firma Rieden als Ausgangspunkt für die Planung aus.
Kurz darauf zeigt ihr das System weitere Informationen entlang der Route und am Ziel auf einer Karte an.
Nachdem sie keine dringenden Termine hat, entschließt sie sich Kunden zu besuchen, bei denen der letzte Besuch schon längerer Zeit zurück liegt (Vorbedingung).
Dafür ändert sie dementsprechend die Einstellungen für die Auswahlkriterien auf die Dauer seit ihrem letzten Besuch, wodurch die Farbcodierungen der Kundenmarker auf der Karte angepasst werden.\\
\\
Kurz darauf kommt ihre Mitarbeiterin Sylvia vorbei und bittet Babsi, ihr ein paar offene Tickets abzunehmen. 
Durch die Umstellung der Filterfunktion werden auf ihrer Karte nun auch zusätzlich die aktuellen Tickets angezeigt.
Sie sieht dass zwei Tickets auf ihrer Route liegen, und öffnet die jeweiligen Kurzinformationen, welche ihr den Titel und die Kurzbeschreibung der einzelnen Tickets anzeigen. 
Da sie beim ersten Ticket schon weiß um was es sich handelt, übernimmt sie es in ihre Auswahlliste.
Die Informationen des zweiten Tickets sagen ihr leider nicht soviel. 
Das ist aber kein Problem, sie ruft direkt die Übersichtsseite des entsprechenden Tickets auf.
In der Übersicht werden ihr alle Informationen zu dem Ticket angezeigt, die sich im System befinden.
Zurück in der Kartenansicht übernimmt sie auch das zweite Ticket. 
Dabei fällt ihr auf, dass in der Nähe des zweiten Tickets noch ein roter Kundenmarker ist.
Sie öffnet die erweiterte Übersicht für den roten Marker und erfährt, dass bei der Firma ZornTec schon seit sieben Monaten keine Betreuung mehr stattgefunden hat.
Als letzten Punkt auf ihrer Planung übernimmt sie noch den Kunden ZornTec in ihre Liste. 
Nun schließt Babsi die Kartenansicht und sieht anhand der Benachrichtigung, dass vom System schon zwei neue interne Tickets für sie angelegt wurden.
In diesen Tickets findet sie, neben der freundlichen Erinnerung einen Termin mit den Ansprechpartnern\_innen der jeweiligen Firmen auszumachen, auch gleich die passenden Kontaktmöglichkeiten von Herrn Müller (Firma ZornTec) und Frau Koch (Firma Rieden).


\section{Abgrenzung}
Wie dieses Kapitel gezeigt hat, bietet der gewählte Themenschwerpunkt ein reichhaltiges Betätigungsfeld, aus dem sich diverse Funktionalitäten ableiten lassen. 
Um das Ziel der Arbeit nicht aus den Augen zu verlieren, wird abschließend noch definiert, welche Eigenschaften für den ersten Entwurf des Prototypen ausgeschlossen werden und dementsprechend auch nicht in dieser Arbeit behandelt werden. 
Dieser Schritt ist notwendig, um eine klare Fokussierung zu setzen sowie Missverständnisse auszuräumen.
Explizit handelt es sich dabei um folgende drei Punkte:\\
\\
\begin{enumerate}
	\item Kein Routenplaner
	\item[] Der Prototyp soll nicht die Routenplanung im herkömmlichen Sinne unterstützen wie dies Navigationssystem oder Webservices wie Beispielsweise Google Maps tun.
	\item Keine Mobile Anwendung
	\item [] Im ersten Schritt ist keine spezielle Implementierung oder Unterstützung für mobile Endgeräte (Smartphones) angedacht. Die Anwendung beschränkt sich ausschließlich auf die Nutzung in einem Webbrowser. Dabei werden folgende Webbrowser unterstützt: Google Chrome, Mozilla Firefox\footnote{Diese Vorgaben werden durch Pery definiert, in welches der Prototyp integriert ist.}
	\item Keine automatische Planung
	\item[] Es ist nicht vorgesehen, dass Anwender\_innen Vorschläge erhalten die vom System erstellt werden. Vielmehr soll die Darstellung der Informationen eine Unterstützung darstellen auf Basis derer Entscheidungen von Anwender\_innen getroffen werden.
\end{enumerate}
\end{document}

