\documentclass[Bachelorarbeit.tex]{subfiles}
\begin{document}
\chapter{Einführung}
\label{chap:einfuehrung}

\section{Idee}
\label{chap:einfuehrung:sec:idee}

Die grundlegende Idee besteht darin, Informationen (Ressourcen und Aufgaben) mit geografischen Daten zu verknüpfen und diese zu visualisieren um somit die Nutzer\_innen bei den Entscheidungsprozessen zu unterstützen. Des Weiteren muss noch definiert werden welche/welcher Anwendungsfall, im Rahmen der Masterarbeit, umgesetzt werden soll.\\
\\
Stichpunkte:
\begin{itemize}
\item (Sinnvolle) Filterung von Daten
\begin{itemize}
	\item Evtl. Zustands- und oder Modus- abhängige Filterung
\end{itemize}
\item Optimierte Darstellungsform
\begin{itemize}
	\item Zielführende Darstellung der verknüpften Informationen
\end{itemize}
\end{itemize}

\section{Hintergrund}
\label{chap:einfuehrung:sec:hintergrund}
Als Grundlage für dieses Projekt dient die bereits entwickelte Software „pery“ der Firma Perfany GmbH welche um die unten genannten Anwendungsfälle erweitert werden soll. Pery ist eine webbasierte (Software as a Service) ERP/CRM Lösung mit dem Schwerpunkt auf vernetzte Informationen. 

\section{Usecase}

\todoImprovement[]{Mehr Storytelling}
Um einen besseres Verständnis für die Umsetzungen zu erlangen wird an der Stelle die Anwendungsszenarien aufgezeigt.



\subsection{Usecase I - Enhanced Ticket (\ac{UC}1)}
Ein wichtiger Bestandteil des (bestehenden) Systems besteht darin Tickets zu verwalten.\footnote{automatische Erstellung, anlegen sowie anderen Mitarbeiter\_innen zuweisen}
Dieses Feature wird verstärkt von \ac{KMU}’s mit Schwerpunkt auf außendienstlichen Tätigkeiten eingesetzt. 
Rückmeldungen von diesen Nutzer\_innen Gruppen hat ergeben, dass der Prozess der Ticket Zuteilung an Mitarbeiter\_innen Optimierungspotential hat. \\
\\

\textbf{Beispiel: IT-Dienstleister}
\begin{enumerate}
	\item Kunde des Dienstleisters erstellt neues Ticket
		\begin{enumerate}
			\item Geo-Daten werden an das Ticket angefügt
		\end{enumerate}
	\item Dispatcher des Dienstleisters reagiert auf Ticket
		\begin{enumerate}
			\item Einstufung der Priorität\footnote{Vorschläge durch das System (Stammdaten - Priorität des verknüpften Kunden) – Auswahl basiert auf der Entscheidung des Dispatcher}
			\item Ressourcen ermitteln:
			\begin{enumerate}
				\item Welche/r Mitarbeiter\_in ist verfügbar und geografisch am nächsten (Anfahrtswegoptimierung)\footnote{Fragestellung: Visualisierung der Ergebnisse }
				\item Ist kein/e Mitarbeiter\_in verfügbar: Vorschläge vom System welcher Mitarbeiter von aktueller Aufgabe abgezogen werden kann (bsp.: interne Aufträge)
			\end{enumerate}
			\item Ticket auf Resource zuweisen\footnote{Resource (Mitarbeiter\_in) und Kunde werden informiert}
		\end{enumerate}
	\item Zugewiesene/r Mitarbeiter\_in hat Ticket gelöst
	\begin{enumerate}
		\item Resourcen wurden vom System auf den Auftrag verbucht
		\item Ticket wird abgeschlossen
	\end{enumerate}
\end{enumerate}

\subsection{Usecase II - On Trip Information (\ac{UC}2)}
Hierbei handelt es sich um weiteres Feature für die Optimierung von Planung- bzw. Arbeitsvorbereitungs- Prozessen von Außendienst Mitarbeitern. Diese sollen bei der Planung ihrer Route, durch das einblenden zusätzlicher Information, unterstützt werden. \\
\\
\textbf{Beispiel: Außendienst Mitarbeiter\_in}
\begin{enumerate}
	\item Mitarbeiter\_in wählt Ziel der Route aus 
	\begin{enumerate}
		\item Ziel kann Ticket, Kunde oder Adresse sein
	\end{enumerate}
	\item System zeigt weitere Informationen entlang der Route oder am Ziel an\footnote{Die Auswahl der Informationen soll gefiltert werden können. Eventuell mehrere Filter Ebenen wie Kundenbetreuung oder offene Tickets. Auf Basis der getroffenen Filterebene können anschließend weitere Filter gewählt werden wie beispielsweise: geplanter Zeitaufwand von offenen Ticket, aktueller Betreuung Status, etc. }
	\begin{enumerate}
		\item Mögliche Informationen:
		\begin{enumerate}
			\item Offene Tickets
			\item Betreuungsstatus von Kunden\footnote{Betreuungstatus: ist ein Schlüssel der sich aus: Betreuungsaufwand, Priorität des Kunden und Dauer seit dem letzten Betreuungstermin zusammensetzt.}
			\item Evtl. weitere Informationen
		\end{enumerate}
	\end{enumerate}
	\item Mitarbeiter\_in wählt zusätzliche Aufgaben aus
	\begin{enumerate}
		\item System weißt das Ticket der/dem Mitarbeiter\_in zu
		\item Evtl. automatisch weitere Tickets anlegen und der/dem Mitarbeiter\_in zuweist.\footnote{Beispiel: Betreuungstermin – System legt automatisch ein Ticket zur Termins-findung/-vereinbarung mit dem Kunden an und weißt es der/dem Mitarbeiter\_in zu.}
	\end{enumerate}
\end{enumerate}

\end{document}