\documentclass[Bachelorarbeit.tex]{subfiles}
\begin{document}
\chapter{Einführung}
\label{chap:einfuehrung}

Das Ziel der Arbeit ist zu untersuchen, 
inwiefern die Ergänzung  von Informationen mit geografischen Daten, 
zu einer Optimierung von Entscheidungen beiträgt und welcher Bedeutung dabei der Darstellungsform zukommt. 
Für diesen Zweck wird, im Zuge der Arbeit, ein Prototyp entwickelt, der die Anwender\_innen bei der Planung von Außendienstrouten unterstützen soll. 
Dabei sollen mit dem Prototypen nicht klassische Probleme der Informatik oder Logistik wie Beispielsweise das "travelings salesman problem" gelöst werden. Vielmehr soll den Anwender\_innen ein Werkzeug zur Verfügung gestellt werden, dass ihnen vernetzte Informationen übersichtlich zur Verfügung stellt um sinnvolle Entscheidungen treffen zu können.

\section{Problemstellung}
\label{chap:einfuehrung:sec:problemstellung}

Bei diversen Gesprächen mit Kunden\_innen trat in regelmäßigen Abständen immer wieder die Nachfrage für eine Softwarelösung, welche die Planung und Organisation von Außendiensttätigkeiten vereinfachen soll auf. 
Was auf den ersten Blick trivial erscheinen mag, wirkt nach den ersten Überlegungen durchaus interessant, beispielsweise in der ersten Phase (Planung) müssen Routen erstellt werden, die in der Praxis aus diversen Datenquellen oder gar unterschiedlichen Medien stammen. 
Des Weiteren fließen in die Planung, kundenspezifische Erfahrungswerte sowie lokale Ortskenntnisse mit ein.
Des Weiteren hält die Außendiensttätigkeit noch nach der Planung weitere interessante Herausforderungen bereit, diese reichen von der Unterstützung bei der Durchführung bis hin zu der Nachbearbeitung.

\begin{comment}
Als Grundlage für den Prototypen dient die Software Pery der Firma Perfany GmbH. 
Dabei handelt es sich um 
Wobei der Fokus auf dem Ticket Modul der Software ruht. Dabei ist das Anwendungskonzept des Moduls so ausgelegt das sämtliche Aufgaben, die die Firma betreffen , einzeln als Tickets erfasst werden. 
\end{comment}

\section{Idee}
\label{chap:einfuehrung:sec:idee}

Basierend auf der \nameref{chap:einfuehrung:sec:problemstellung} soll sich an dieser Stelle platz für die ersten eigenen Überlegungen finden. 
Diese Ideen stellen noch keine endgültige Lösung da, sondern werden progressiv durch kommende Analysen, \nameref{chap:analyse:sec:interviews} und Feedbacks angepasst.\\
\\
Die grundlegende Idee besteht darin, Informationen (Ressourcen und Aufgaben) mit geografischen Daten zu verknüpfen und diese zu visualisieren um somit die Nutzer\_innen bei den Entscheidungsprozessen zu unterstützen. 
Dabei soll auf die beiden Bereiche: ``Datenfilterung/-Anreicherung'' sowie ``Darstellungsformen'' besonderes Augenmerk gelegt werden.

\paragraph*{Datenfilterung/-Anreicherung}
Ein wichtiger Aspekt, für die maximale Unterstützung durch den Prototypen bei der Planung besteht darin, dass die richtigen Informationen, zum richtigen Zeitpunkt am richtigen Ort vorhanden sind. 
Ein Beispiel für ein sehr wahrscheinlich auftretende Datenanreicherung ist, wie zuvor erwähnt, die Verwendung von geographischen Informationen. 
Inwiefern es zu dem Filtern beziehungsweise Anreichern der Daten kommt wird sich Anhand der Analyse der Bedürfnisse (siehe Kapitel.: \ref{chap:analyse:sec:interviews} - \nameref{chap:analyse:sec:interviews}) herausstellen.

\paragraph*{Darstellungsformen}
Die Visualisierung der Daten steht in keinster Weise in einer untergeordneten Rolle.
Erst durch den sinnvollen Einsatz der Darstellung, werden die vorhandenen Daten zu einem wichtigen Indikator bei der Entscheidungsfindung.
Dabei versucht die Arbeit sich kritisch mit der Optimierung der Darstellung auseinander zusetzen und diese in späteren Versuchen am Prototyp zu evaluieren.
Ein Beispiel dafür wäre die Kartenansicht.
Viele Anwendungen setzen Kartenansichten bei der Darstellung von geografischen Daten ein. 
Diese Arbeit wird sich unter anderem mit der Frage auseinander setzen, welche Planungs-Szenarios beziehungsweise Workflow-Schritte durch eine Karten- und/oder Listenansicht besser unterstützt werden.
Um dies herauszufinden wird, im späteren Verlauf, eine Analyse mit Hilfe des Prototypen durchgeführt.
Eine weitere Überlegung besteht darin, die Anwender\_innen selbst entscheiden zu lassen, welche Ansicht sie für welchen Zweck bevorzugen.


\section{Hintergrund}
\label{chap:einfuehrung:sec:hintergrund}
Als Grundlage für dieses Projekt dient die bereits entwickelte Software „pery“ der Firma Perfany GmbH welche um die unten genannten Anwendungsfälle erweitert werden soll. Pery ist eine webbasierte (Software as a Service) ERP/CRM Lösung mit dem Schwerpunkt auf vernetzte Informationen. 

\section{Usecase}

\todoImprovement[]{Mehr Storytelling}
Um einen besseres Verständnis für die Umsetzungen zu erlangen wird an der Stelle die Anwendungsszenarien aufgezeigt.



\subsection{Usecase I - Enhanced Ticket (\ac{UC}1)}
Ein wichtiger Bestandteil des (bestehenden) Systems besteht darin Tickets zu verwalten.\footnote{automatische Erstellung, anlegen sowie anderen Mitarbeiter\_innen zuweisen}
Dieses Feature wird verstärkt von \ac{KMU}’s mit Schwerpunkt auf außendienstlichen Tätigkeiten eingesetzt. 
Rückmeldungen von diesen Nutzer\_innen Gruppen hat ergeben, dass der Prozess der Ticket Zuteilung an Mitarbeiter\_innen Optimierungspotential hat. \\
\\

\textbf{Beispiel: IT-Dienstleister}
\begin{enumerate}
	\item Kunde des Dienstleisters erstellt neues Ticket
		\begin{enumerate}
			\item Geo-Daten werden an das Ticket angefügt
		\end{enumerate}
	\item Dispatcher des Dienstleisters reagiert auf Ticket
		\begin{enumerate}
			\item Einstufung der Priorität\footnote{Vorschläge durch das System (Stammdaten - Priorität des verknüpften Kunden) – Auswahl basiert auf der Entscheidung des Dispatcher}
			\item Ressourcen ermitteln:
			\begin{enumerate}
				\item Welche/r Mitarbeiter\_in ist verfügbar und geografisch am nächsten (Anfahrtswegoptimierung)\footnote{Fragestellung: Visualisierung der Ergebnisse }
				\item Ist kein/e Mitarbeiter\_in verfügbar: Vorschläge vom System welcher Mitarbeiter von aktueller Aufgabe abgezogen werden kann (bsp.: interne Aufträge)
			\end{enumerate}
			\item Ticket auf Resource zuweisen\footnote{Resource (Mitarbeiter\_in) und Kunde werden informiert}
		\end{enumerate}
	\item Zugewiesene/r Mitarbeiter\_in hat Ticket gelöst
	\begin{enumerate}
		\item Resourcen wurden vom System auf den Auftrag verbucht
		\item Ticket wird abgeschlossen
	\end{enumerate}
\end{enumerate}

\subsection{Usecase II - On Trip Information (\ac{UC}2)}
Hierbei handelt es sich um weiteres Feature für die Optimierung von Planung- bzw. Arbeitsvorbereitungs- Prozessen von Außendienst Mitarbeitern. Diese sollen bei der Planung ihrer Route, durch das einblenden zusätzlicher Information, unterstützt werden. \\
\\
\textbf{Beispiel: Außendienst Mitarbeiter\_in}
\begin{enumerate}
	\item Mitarbeiter\_in wählt Ziel der Route aus 
	\begin{enumerate}
		\item Ziel kann Ticket, Kunde oder Adresse sein
	\end{enumerate}
	\item System zeigt weitere Informationen entlang der Route oder am Ziel an\footnote{Die Auswahl der Informationen soll gefiltert werden können. Eventuell mehrere Filter Ebenen wie Kundenbetreuung oder offene Tickets. Auf Basis der getroffenen Filterebene können anschließend weitere Filter gewählt werden wie beispielsweise: geplanter Zeitaufwand von offenen Ticket, aktueller Betreuung Status, etc. }
	\begin{enumerate}
		\item Mögliche Informationen:
		\begin{enumerate}
			\item Offene Tickets
			\item Betreuungsstatus von Kunden\footnote{Betreuungstatus: ist ein Schlüssel der sich aus: Betreuungsaufwand, Priorität des Kunden und Dauer seit dem letzten Betreuungstermin zusammensetzt.}
			\item Evtl. weitere Informationen
		\end{enumerate}
	\end{enumerate}
	\item Mitarbeiter\_in wählt zusätzliche Aufgaben aus
	\begin{enumerate}
		\item System weißt das Ticket der/dem Mitarbeiter\_in zu
		\item Evtl. automatisch weitere Tickets anlegen und der/dem Mitarbeiter\_in zuweist.\footnote{Beispiel: Betreuungstermin – System legt automatisch ein Ticket zur Termins-findung/-vereinbarung mit dem Kunden an und weißt es der/dem Mitarbeiter\_in zu.}
	\end{enumerate}
\end{enumerate}

\end{document}