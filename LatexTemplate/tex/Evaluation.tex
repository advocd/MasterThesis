\documentclass[Bachelorarbeit.tex]{subfiles}
\begin{document}
\chapter{Evaluation}
\label{chap:evalutation}

\ideas{Einleitung und Fragestellung -- These und Nullthese (keine Verbesserung), Usability-Analyse (basierende auf ISO...: effektiv: konnte Problem lösen; effizient: wie schnell wurde das Problem gelöst (Dauer linear zu Schwierigkeitsgrad + Selbsteinschätzung im Fragebogen); Zufriedenheit: wie erkundbar ist das UI?)}

\section{Methodik}

\ideas{Kontext = das Szenario (Kunden aussuchen auf Basis von Kriterien); Zielgruppe}

\subsection{Verfahren}

\ideas{Eyetracking und Fragebogen}

\subsection{Stichprobenbeschreibung}

\ideas{Wie wurden die Personen ausgesucht}

\subsection{Aufgabenstellung}

\ideas{Beschreibung und Ausarbeitung der Aufgabenstellung -- Sowie die Überlegungen dahinter. Steigender Schwierigkeitsgrad}

\section{Ergebnisse}
\ideas{nüchtern und ohne Interpretation}

\section{Interpretation \& Diskussion}




\end{document}