\documentclass[Bachelorarbeit.tex]{subfiles}
\begin{document}
\chapter{Evaluation}
\label{chap:evalutation}

\ideas{Einleitung und Fragestellung -- These und Nullthese (keine Verbesserung), Usability-Analyse (basierende auf ISO...: effektiv: konnte Problem lösen; effizient: wie schnell wurde das Problem gelöst (Dauer linear zu Schwierigkeitsgrad + Selbsteinschätzung im Fragebogen); Zufriedenheit: wie erkundbar ist das UI?)}



Nachdem der Prototyp implementiert wurde, stellt sich die Frage, ob die Entwicklung und Ausarbeitung einen Mehrwert für Anwender\_innen darstellt. 
Dies soll nun mittels der Evaluation geklärt werden.
Für diesen Zweck sowie etwaige Schwachstellen aufzufinden soll der Prototyp im Rahmen einer Usability-Analyse auf Effektivität, Effizienz und Zufriedenstellung \cite[vgl.][Abs.: 3]{Iso9241_11} untersucht werden (siehe Absatz: \nameref{Methodik}).
Anschließend werden die Ergebnisse der Analyse präsentiert.
Abgeschlossen wird das Kapitel durch die Interpretation und Diskussion.


\paragraph{Nullhypothese (H0)}
Der Prototyp unterstützt die Anwender\_innen minimal oder nicht messbar in den Bereichen Effektivität, Effizienz und Zufriedenstellung (Die Begriffe Effektivität, Effizienz und Zufriedenstellung beziehen sich auf die Definition nach der Norm EN ISO 9241-11, \cite[vgl.][Abs.: 3]{Iso9241_11}).

\paragraph{Alternativehypothese (H1)}
Der Prototyp unterstützt die Anwender\_innen messbar in den Bereichen Effektivität, Effizienz und Zufriedenstellung (Die Begriffe Effektivität, Effizienz und Zufriedenstellung beziehen sich auf die Definition nach der Norm EN ISO 9241-11, \cite[vgl.][Abs.: 3]{Iso9241_11}).


\section{Methodik}
\label{Methodik}
\ideas{Kontext = das Szenario (Kunden aussuchen auf Basis von Kriterien); Zielgruppe}

Da die Interviews im Vorfeld ergeben haben, dass jede befragte Person einen unterschiedlichen Ablauf sowie unterschidliche Werkzuege bei der Plannung der Aussendienstrouten einsetzt macht ein A/B-Test nicht sehr viel Sinn.
Aus dem Grund viel die Entscheidung, hier einen klaren Schnitt zu den alten Systemen zu ziehen und einen allgemeinen Usability Test nach ISO XXX durchzuführen





\subsection{Verfahren}

\ideas{Eyetracking und Fragebogen}

\subsection{Stichprobenbeschreibung}

\ideas{Wie wurden die Personen ausgesucht}

\subsection{Aufgabenstellung}

\ideas{Beschreibung und Ausarbeitung der Aufgabenstellung -- Sowie die Überlegungen dahinter. Steigender Schwierigkeitsgrad}

\section{Ergebnisse}
\label{Ergebnisse}
\ideas{nüchtern und ohne Interpretation}

\section{Interpretation \& Diskussion}
\label{InterpretationDiskussion}



\end{document}