\documentclass[Bachelorarbeit.tex]{subfiles}
\begin{document}
\chapter*{Zusammenfassung}

In der aktuellen \texttt{Post-PC-Ära} verschiebt sich der Fokus der Endanwender immer mehr von statischen \texttt{Desktops} hin zu mobilen Geräten. 
Der fortgeschrittenen technischen Entwicklung sowie dem großflächigen Ausbau des Mobilfunknetzes ist es zu verdanken, dass  sich diese leistungsstarken Lösungen ihren festen Platz in der Unterhaltungs- und Informationsindustrie gesichert haben.
Das Einsatzgebiet der mobilen Applikationen reicht von kleineren Spielereien über nützliche Hilfsmittel im Alltag bis hin zu Business-Applikationen.
Für Anbieter von digitalen Inhalten und Dienstleistungen ist die eigene mobile Applikation zum Aushängeschild ihres Unternehmens geworden.
Mit der hohen Nachfrage nach mobilen Geräten ist auch der Wettbewerb unter den Herstellern gestiegen. 
Dies führte wiederum zu einem dynamischen Markt der zur Verfügung stehenden Plattformen\\
\\
Um bei der Verbreitung von Apps einen möglichst großen Kundenkreis zu erschließen ist es notwendig, dass die Applikationen von mehr als einer der großen Plattformen unterstützt wird.
Diese Arbeit beschäftigt sich mit der Herausforderung sowie den damit verbundenen Möglichkeiten, eine Applikation für die mobilen Plattformen \texttt{Android} und \texttt{Windows Phone 8} zu entwickeln.
Für diesen Zweck wurde - in Kooperation mit der \ac{OJAD} - ein Anforderungs-Profil ausgearbeitet.
Der definierte Aufgabenbereich des Projektes liegt in der mobilen Unterstützung der SozialarbeiterInnen bei der Dokumentation von Veranstaltungen.\\
\\
Ausgehend von den Projekt-\nameref{sec:anforderungen}, wird ein Kriterien-Katalog für die \nameref{chap:state_of_the_art}-Analyse erstellt.
Anhand dieser Kriterien wird eine Analyse über die aktuellen Entwicklungs-Ansätze sowie jeweils ein stellvertretendes Framework durchgeführt. 
Mit der Auswertung der Ergebnisse wird die State of the Art-Analyse abgeschlossen.
%Den Abschluss der Analyse bildet dabei die \nameref{sec:auswertung} der Ergebnisse. 
In besagter Auswertung ist definiert, dass die Applikation auf der Basis eines \texttt{nativen Ansatzes} umgesetzt werden soll.\\
Die eigentliche Realisierung der Beispiel-Applikation ist in die beiden Kapitel \texttt{\nameref{chap:entwicklung}} und \texttt{\nameref{chap:implementierung}} aufgeteilt.
Der Inhalt des Kapitels \nameref{chap:entwicklung} beschäftigt sich mit der generellen \nameref{sec:architektur}, der Planung des \ac{UI}-Konzeptes sowie dem Aufbau der Programmlogik.
Auf die Einzelheiten der jeweiligen \ac{SDK}'s sowie ihre Verwendung in der Beispiel-Applikation im Kapitel:  \nameref{chap:implementierung} eingegangen. 
Den Abschluss dieser Arbeit bildet eine Zusammenfassung über die Erkenntnisse und Anmerkungen, die während des Entwicklungsprozesses dieser Arbeit sowie der Beispiel-Applikation zustande gekommen sind.




\begin{comment}
Die Arbeitsdokumentation stellt für Organisationen einen wichtigen Faktor bei der Qualitätssicherung dar. 
Um aussagekräftig Analysen erstellen zu können, müssen die gesammelten Daten korrekt und konsistent sein. 
Leider zeigt die Erfahrung,  dass - vor allem im Umfeld von Vereinen und gemeinnützigen Organisationen – nur selten einheitlichen Methoden zum Erfassen vorhanden sind und/oder diese nicht Sachgemäß verwendet werden. 
In Zeiten von Smartphones liegt der Gedanke nahe, die aktuelle Technologie zu verwenden um den Arbeitsaufwand der Dokumentation zu minimieren. \\
\\
Damit eine breite Masse als Zielgruppe angesprochen werden kann, sollte das Programm für die geläufigsten mobilen Betriebssystem verfügbar sein. 
Im Moment besitzen die beiden konkurrierenden Betriebssystemen Android und iOS die größten Marktanteile im deutschsprachigen Raum. 
Allerdings versucht Microsoft, mit Windows Mobile, immer aggressiver Marktanteile zu gewinnen. 
Da diese Plattformen grundsätzlich nicht miteinander kompatibel sind, muss die Applikation für jedes Betriebssystem eigens implementiert werden. 
Die Nachteile liegen auf der Hand, neben dem Entwicklungsaufwand wird auch die Wart- und Erweiterbarkeit eingeschränkt.  \\
\\
Um diese Problematik zu entschärfen, gibt es verschiedene Lösungsansätze. 
Zum einen kann das Programm in zwei Teile Aufgeteilt werden. 
Bei diesem Vorgehen besteht die Applikation aus einen plattformunabhängigen- und einen plattformspezifischen Teil. 
Hierbei liegt das Ziel darin, den plattformspezifischen Code so weit wie möglich zu minimieren. Wodurch der Grad der Wiederverwertbarkeit für andere Plattformen gesteigert wird. 
Zum anderen existieren verschiedene Frameworks für die Multiplattform-Entwicklung. 
Durch den Einsatz dieser Frameworks soll das Programm nur einmal für alle unterstützen Plattformen entwickelt werden. Bei diesem Vorgehen programmiert der Entwickler gegen die plattformunabhängigen Schnittstellen des jeweiligen Frameworks.  \\
\\
Grundsätzlich haben beide Ansätze Stärken und Schwächen, die je nach den Bedürfnissen des Projektes anders zu gewichten sind. 
Das Ziel dieser Arbeit ist es, die verschiedenen Möglichkeiten zu analysieren und anhand einer Beispiel-Applikation für die Plattformen Android und Windows Phone8 zu implementieren.
\end{comment}

\end{document}