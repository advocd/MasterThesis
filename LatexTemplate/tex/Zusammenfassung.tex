\documentclass[Bachelorarbeit.tex]{subfiles}
\begin{document}
\chapter*{Kurzfassung}
Bei Entscheidungsfindungsprozessen werden verschiedenste Informationen aus unterschiedlichen Quellen zusammengetragen, deren Qualität und Darstellung maßgeblich für den Prozess ist.
Das Ziel der vorliegenden Arbeit ist zu untersuchen, inwiefern die Ergänzung von Informationen mit 
geografischen Daten zu einer Optimierung bei der Entscheidungsfindung beiträgt und welche Bedeutung 
dabei der Darstellungsform (Listen- oder Kartenansicht) zukommt.
\\
\\
Für diesen Zweck wird im Zuge der Arbeit ein Prototyp entwickelt, der Anwender\_innen bei der Planung von Außendienstrouten unterstützen soll. 
Mit dem Prototypen sollen nicht klassische Probleme der Informatik oder Logistik, wie beispielsweise das Problem des Handlungsreisenden gelöst werden. Vielmehr soll den Anwender\_innen ein Werkzeug zur Verfügung gestellt werden, welches vernetzte Informationen übersichtlich visualisiert, um wirtschaftliche Entscheidungen treffen zu können.
Um den Prototypen möglichst nahe an den Wünschen und Bedürfnissen der Anwender\_innen zu entwickeln, stützt sich das Konzept auf die Ergebnisse der durchgeführten Interviews mit Domänenexpert\_innen.
\\
\\
Für die Evaluation des Prototypen wurde eine Gebrauchstauglichkeitsanalyse nach ISO 9241-11 durchgeführt. Anwender\_innen mit unterschiedlichem Vorwissen wurden gebeten Aufgaben aus dem Bereich der Außendienstplanung mithilfe des Prototypen zu bewältigen. Während des Tests wurde eine Eyetracking-Analyse durchgeführt und im Anschluss an den Test wurden die Testpersonen mittels eines Fragebogens über die ihre Anwendungserfahrung befragt.
\\
\\
Das Ergebnis der Gebrauchstauglichkeitsanalyse war positiv - alle Testpersonen waren in der Lage die ihnen gestellten Aufgaben erfolgreich zu lösen und die subjektiven Bewertungen liegen im positiven Bereich.
Wie die Auswertung ergeben hat wird mit steigender Komplexität auch häufiger auf die ergänzende Kartenansicht zurückgegriffen. 
Das äußert sich einerseits in einer längeren Verwendungsdauer der Kartenansicht und andererseits durch häufigere Wechsel zwischen den beiden Ansichten. 
\\
\\
Damit erweist sich die Kartenansicht als eine sinnvolle Ergänzung zur konservativen Listenansicht, ersetzt diese aber keineswegs vollständig.
\end{document}