\documentclass[Bachelorarbeit.tex]{subfiles}
\begin{document}
\chapter*{Zusammenfassung}

Das Ziel der Arbeit ist zu untersuchen, 
inwiefern die Ergänzung  von Informationen mit geografischen Daten, 
zu einer Optimierung von Entscheidungen beiträgt und welche Bedeutung dabei der Darstellungsform zukommt. 
Für diesen Zweck wird im Zuge der Arbeit ein Prototyp entwickelt, der die Anwender\_innen bei der Planung von Außendienstrouten unterstützen soll. 
Dabei sollen mit dem Prototypen nicht klassische Probleme der Informatik oder Logistik wie beispielsweise das Traveling Salesman Problem gelöst werden. Vielmehr soll den Anwender\_innen ein Werkzeug zur Verfügung gestellt werden, welches ihnen vernetzte Informationen übersichtlich zur Verfügung stellt, um wirtschaftliche Entscheidungen treffen zu können.
Um den Prototypen möglichst nahe an den Wünschen und Bedürfnissen der Anwender\_innen zu entwickeln stützt sich das Konzept auf die Ergebnisse der durchgeführten Interviews mit den\\
\\
Für die Evaluation wurde ein Eyetracking-Test durchgeführt bei dem die Testpersonen verschiedene 

\end{document}