\documentclass[Bachelorarbeit.tex]{subfiles}
\begin{document}
\chapter{Diskussion \& Reflexion}
\label{chap:reflexion}


\section{Zusammenfassung}
\label{chap:reflexion:sec:zusammenfassung}

Generalisung  der Ergebnisse

\ideas{Verbesserung durch die Arbeit - Was hat die Arbeit der Welt gebracht -- Daten zusammenführen und Darstellung}
- Entscheidungsfehler
- Sowohl Karten als Listenansicht
- Redesign der Listenansicht - add button
\ideas{Resümee... was ist gut was ist schlecht gelaufen, was würde ich anders machen. \\
- Auswahl des Webmapping
}

\section{Ausblick}
\label{chap:reflexion:sec:ausblick}

\ideas{further research, etc. kommt hier rein. Evtl. 3D Darstellung, Anreichern vs. Filtern, Punkte clustern}
- siehe Fragebogen
- Suchfunktion
- Ausbauen zum Framework
\begin{comment}
	Überlegenswert wäre auch der Einsatz aus der Kombination 3D Modellen und eines Graphens. 
	Dadurch erweitert sich die Darstellung aus dem zwei Dimmensinalen Raum (Karte) hin zu einem dreidimmensionalen Raum in dem Informationen genauer abgebildet werden können\\
	\\
	/aufwand\\
	/nutzen
\end{comment}

\end{document}