\documentclass[Bachelorarbeit.tex]{subfiles}
\begin{document}
\chapter{Diskussion \& Reflexion}
\label{chap:reflexion}

\label{chap:reflexion:sec:zusammenfassung}

\ideas{Interviews: mehr Personen anderer Modus?}
\ideas{Evaluation Gruppeneinteilung, Modus}
\ideas{Leaflet.js}


Das Ziel dieses Kapitels liegt darin die Erkenntnisse zusammenzufassen und die Vorgehensweise kritisch zu hinterfragen.
Für diesen Zweck wird noch einmal ein Blick auf den Entstehungsprozess und die Motivation des Prototypen geworfen.
Darauf folgt ein Überblick der Evaluation und über welchen Mehrwert der Prototyp, im Hinblick auf das Anwendungsszenario verfügt.
Ergänzend dazu werden anschließend die gewonnen Erkenntnisse im Einsatz der Karten- und Listenansicht in Kombination mit kontextsensitiven Informationen aufgezeigt.
Abgeschlossen wird das Kapitel durch einen Ausblick auf die weiteren Entwicklungsmöglichkeiten und Potentiale des Prototypen. 

\subsection*{Motivation und erste Überlegungen}
Die Motivation dieser Arbeit besteht darin die Planung bei der Ressourceneinteilung zu optimieren, konkret an dem Beispiel der Außendienstplanung. 
Auf Grund von Gesprächen erfuhr ich wie Umständlich die Planung von Außendiensttätigkeiten, mit den zur Verfügung stehenden Mitteln ist und setzte mir das Ziel zu untersuchen wie diese Arbeit erleichtert und optimiert werden kann.\\
\\
Anhand erster Überlegungen wurde eine Problemstellung aufgesetzt welche für den weiteren Vorgang der Arbeit als Grundlage dienen soll. 
Dabei muss beachtet werden das es sich bei der Problemstellung (siehe Gleichnamigen Abschnitt im Kapitel Einführung) um reine Annahmen handelt, da der Autor keine eigene Erfahrungen auf dem Gebiet der Außendienstplanung besitzt.
Im speziellen wurden dabei die Themen verteilte Informationen, Komplexität und Wissensmanagement identifiziert (siehe Abschnitt \nameref{chap:einfuehrung:sec:problemstellung})
Mittels der Annahmen aus der Problemstellung wurden erste Lösungsansätze entwickelt um zu erfahren ob und wie diese hypothetischen Problem lösbar sind.
Dafür wurde unter anderem überlegt der Komplexität mit Hilfe von verschiedenen Visualisierungsformen (Karten- und Listenansicht) entgegenzutreten.
Auf Basis dieser Erkenntnisse wurde ein mögliches Anwendungsszenario entworfen wie ein fiktiver Prototyp verwendet werden könnte.

\subsection*{Überblick der bestehenden Konzepte}
Darauf hin wurde untersucht wie etablierte Webseiten, mit einem ähnlichen Aufgabengebiet, ihre Informationen visualisieren und was sie den Nutzer\_innen für Möglichkeiten bieten (siehe Kapitel \nameref{chap:analyse}). 
Zusätzlich zu den bestehenden Lösungen wurden technischen Möglichkeiten betrachtet mit dessen Hilfe sich ein Prototyp realisieren lässt.
Anhand der Bedarfsanforderung (siehe gleichnamigen Abschnitt) wurde für diesen Zweck Leaflet.js, aufgrund der Abdeckung der Anforderung sowie seiner einfachen Verwendungsweise und, ausgewählt (siehe Abschnitt \nameref{AuswahlDerTechnologie}).

\subsection*{Entwurf eines eigenen Konzeptes}
Anhand der erarbeiteten Erkenntnisse wurden gezielte Interviews mit Fachpersonen aus dem Außendienst durchgeführt um zu erfahren wie ihre Aufgabenstellungen im Alltag aussehen und auf welche Weise und mit welchen Werkzeugen sie diese bewältigen (siehe Abschnitt \nameref{chap:analyse:sec:interviews}). 
Anhand der Ergebnisse aus den Interviews wurden die zuvor getroffenen Annahmen größtenteils bestätigt (siehe Abschnitt \nameref{AnalyseInterviews}).
Auf der Grundlage des neuen Wissensstandes wurde das Konzept für den ersten Prototypen konkretisiert. 
Als Schwerpunkte wurden dabei die Auflösung der Systembrüche und die Darstellung von kontextsensitiven Daten in verschiedenen Arten definiert.
Diese sollen durch die Integration des Prototypen in die bestehende Softwarelösung Pery gelöst werden. \\
\\
Mittels der bestehenden \ac{ERP}/\ac{CRM} Funktionalität von Pery sollen nicht nur die bestehenden Daten sondern, anhand des Kontextes, auch zusätzliche Informationen in verschiedenen Ansichten visualisiert werden um den Entscheidungsfindungsprozess zu vereinfachen. 

\subsection*{Realisierung des Prototyps}

Für diesen Zweck wird eine Problemstellung 
Für diesen Zweck wird untersucht ob sich die    
/ was bringt der Prototyp\\
\\
/ reflexion des Prozesses\\
\\
/ was bringt der Prototyp\\
\\
/ Was bringen die Ergebnisse unabhängig des Prototypes\\


Leider konnte im Zuge dieser Arbeit keine Allumfassende Lösung Entwickelt werden die alle Bedürfnisse 

\ideas{Verbesserung durch die Arbeit - Was hat die Arbeit der Welt gebracht -- Daten zusammenführen und Darstellung}
- Entscheidungsfehler
- Sowohl Karten als Listenansicht
- Redesign der Listenansicht - add button
\ideas{Resümee... was ist gut was ist schlecht gelaufen, was würde ich anders machen. \\
- Auswahl des Webmapping
}

\section{Ausblick}
\label{chap:reflexion:sec:ausblick}

\ideas{further research, etc. kommt hier rein. Evtl. 3D Darstellung, Anreichern vs. Filtern, Punkte clustern}
- siehe Fragebogen
- Suchfunktion
- Ausbauen zum Framework
\begin{comment}
	Überlegenswert wäre auch der Einsatz aus der Kombination 3D Modellen und eines Graphens. 
	Dadurch erweitert sich die Darstellung aus dem zwei Dimmensinalen Raum (Karte) hin zu einem dreidimmensionalen Raum in dem Informationen genauer abgebildet werden können\\
	\\
	/aufwand\\
	/nutzen
\end{comment}

\end{document}