\documentclass[Bachelorarbeit.tex]{subfiles}
\begin{document}
\chapter*{Abstract}
In decision-making various information from various sources is gathered and combined, the quality and visualization of which essentially impacts the process.
The main objective of this thesis is to analyze how supplementing information with geographical data improves decision-making and what impact the form of presentation has.
\\
\\
In the course of this thesis the author developed a prototype that supports users in planning field services. 
The objective of the prototype is not to solve typical computer science and logistics problems such as the Travelling Salesman Problem - it is rather to provide users with a tool that visualizes interconnected information in a neat manner, to support them in making economic decisions.
To ensure that the prototype fulfills as many of the needs and wishes of users as possible, the concept relies on the results of interviews with domain experts.
\\
\\
The evaluation of the prototype was conducted with a usability analysis as defined by ISO-9241-11, in which users with varying preknowledge were asked to solve tasks in planning field services.
This test was accompanied by an eyetracking-analysis, after which the test subjects were asked to complete a questionnaire regarding their user experience, focussing on efficiency, effectiveness and satisfaction.
\\
\\
The result of the usability analysis was positive, as all of the subjects were able to successfully solve the given tasks and gave positive subjective reviews.
The evaluation revealed that with increasing complexity of the tasks users increasingly resorted to the complementary map view.  
This expressed itself in a longer period of use of the map view, as well as an increasing number of shifts between both views.
\\
\\
In conclusion the map view proves to be a valuable addition to the preexisting list view, however, it does not fully replace it.

\end{document}