\documentclass[Bachelorarbeit.tex]{subfiles}
\begin{document}
\chapter*{Abstract}

In the current post PC era the end users focus shifts from static desktops to mobile devices.
It’s due to the advanced technical development and the extensive expansion of mobile phone networks, these powerful solutions have secured their place in the entertainment and information industry.
Mobile applications can range from small shenanigans over useful helpers in everyday life to critical business applicaitons.
Providers of digital services their own mobile application is the flagship to communicate their corporate identity.
With the increased demand of mobile devices competition in this market has been heightened, which leaded to a dynamic market of available platforms.\\
\\
To reach the widest possible range of customers for apps it is indespensable to support more than one significant platform.
This work is concerned with this challenge, as well as the associated possibilities, to develop an application for the mobile platforms Android and Windows Phone 8.
For this purpose a demand profile has been developed in cooperation with OJAD. The defined task range of the project is the mobile support of social workers with the documentation of events.\\
\\
By means of the defined project requirements I created a criteria catalogue for the State of the Art analysis, based on which current development approaches and representative frameworks have been studied.
With the evaluation of the according results the State of the Arts analysis will be completed. According to the given results the application will be developed in a native approach.
The realization of the project is split into two chapters, which are Entwicklung (Development) and Implementierung (Implementation). 
The chapter Entwicklung deals with the architecture of the application, the UI concept and the structure of the application logic. Details about the regarded SDKs, as well as their usage in an example application and peculiarities will be discussed in the chapter Implementierung.
The final part of the work is a conclusion about the insights and annotations that came up in the process of developing this work and the example application.

\end{document}