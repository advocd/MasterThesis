\documentclass[Bachelorarbeit.tex]{subfiles}
\begin{document}
\chapter{Entwicklung}
\label{chap:entwicklung}


\begin{comment}


\section{Spezifikation}

\begin{itemize} \color{red}
\item Beschreibung welche Technologien eingesetzt werden

\end{itemize}

\end{comment}

\section{Entwicklungsspezifikationen}
\label{sec:usecases}


\section{Design-Entwurf}
\label{sec:design_entwurf}


\subsection*{Ziele der Gestaltung}
\label{sub:ziele_der_gestaltung}
Um die Ziele der Gestaltung definieren zu können muss zuerst analysiert werden, wie die Zielgruppe ihre Wünsche und Erwartungen an die Applikation definiert.\\

\subsubsection*{Ziel: Übersichtlichkeit}
\label{subsub:ziel_uebersichtlichkeit}
Eine klare und aufgeräumte Oberfläche ist notwendig, um die Unterstützung der Benutzer zu maximieren. 
Dazu zählt das Aufteilen der Inhalte in sinnvolle und logische Gruppen unter der Berücksichtigung der Arbeitsabläufe.


\subsubsection*{Ziel: Vertraute Umgebung}
\label{subsub:ziel_vertraute_umgebung}
Durch das Erzeugen einer vertrauten Umgebung sollen die Nutzer bekannte Elemente ihrer Plattform in der Applikation wiedererkennen.
Dadurch sollen die Einarbeitungszeiten in die Applikation minimiert, sowie der Wiedererkennungswert von plattformtypischen Elementen wie beispielsweise die Navigation durch das Programm, maximiert werden.
Es wird versucht, dass die Benutzer jeder Zeit das Gefühl haben, dass sie das Programm vollständig kontrollieren.
Dadurch soll den Nutzern die Angst genommen werden "`etwas falsches zu machen"'.
Für die Realisierung dieses Zieles muss sich die Applikation an dem Standartverhalten der jeweiligen Plattform orientieren.
\newpage

\subsection*{Mock-Ups - Prototyp Entwicklung}
\label{sub:mockUps}
Um schon in einem frühen Stadium des Entwicklungsprozesses\\ Rückmeldungen über die Bedienbarkeit des Programmes zu erhalten, empfiehlt es sich, in die Entwicklung von Mock-ups zu investieren.
Unter Mock-Ups versteht man einen nicht funktionellen Prototypen. 
Dies bedeutet, dass der Prototyp nicht programmiert, sondern mit einem Bildbearbeitungsprogramm erzeugt wird.
Dabei wird für jede Ansicht ein eigener Prototyp "`gezeichnet"'. 
Anschließend werden mit Personen aus der Zielgruppe die einzelnen Szenarios der Applikation simuliert. 
Während des Testes sollen die NutzerInnen den Ablauf des Programmes, die erwarteten ihrer Ergebnis von Interaktionen sowie ihre Emotionen so detailliert wie möglich kommentieren.  
Mithilfe des Testes lässt sich feststellen, wie schlüssig die Arbeitsabläufe sind, ob die Funktionalität der Oberfläche verständlich ist und ob die Reaktionen des Programmes mit den Erwartungen der Benutzer übereinstimmen. 
Der Prozess des \texttt{Paper Prototyping} sollte nicht einmalig, sondern in Form eines iterativen Entwicklungsprozess durchgeführt werden. 
Dies bedeutet, dass auf der Basis der erhaltenen Rückmeldungen die Mock-Ups optimiert und die Test wiederholt werden.
Der Test kann als positiv betrachtet werden, wenn die Rückmeldungen der Probanden mit den zuvor definierten Design-Zielen (siehe Abschnitt: \nameref{sub:ziele_der_gestaltung}) übereinstimmen.\\
\\
Die Mock-Ups der Beispiel-Applikationen befinden sich im Anhang:  \nameref{chap:diagramme_und_bilder} unter dem Abschnitt: \nameref{sec:mock_ups}. 

\newpage

\section{Architektur}
\label{sec:architektur}


\end{document}
